\documentclass{article}
\input{C:/Users/khali/OneDrive/AUS/Classes/6_F23/preamble.tex}

\geometry{a4paper, top=1in, bottom=1in, left=1in, right=1in, twoside}

\newcommand{\chpline}{\par\noindent\rule[1cm]{\textwidth}{0.4pt}}
\newcommand{\qline}{\par\noindent\rule{4.5in}{1pt}}
\renewcommand{\sectionmark}[1]{\markboth{#1}{}}

\titleformat{\section}{\normalfont\LARGE\bfseries}{}{0pt}{}

\renewcommand{\contentsname}{\LARGE Contents}

\pagestyle{fancy}
\fancyhf{}
\fancyhead[LE,RO]{\thepage}
\fancyhead[RE,LO]{\large \textit{\leftmark}}

\hypersetup{
	colorlinks=true,
	linkcolor=blue,
	filecolor=magenta,
	urlcolor=cyan,
	pdftitle={MTH 350 - Actuary Problem Set},
	pdfpagemode=FullScreen,
}

\setlength{\droptitle}{8cm}

\title{Actuary Problems Solution Set \\
	\Large Probability for Risk Management}
\author{\\Khalifa Salem Almatrooshi \\
	American University of Sharjah}
\date{November 2023}

\begin{document}

	\begin{titlepage}
		\maketitle
	\end{titlepage}

	\tableofcontents

	\clearpage


	\section[Chapter 2: Counting for Probability]{Chapter 2 \\
		Counting for Probability}

		\subsection*{2-47}

		\begin{figure}[!h]
			\centering
			\begin{tikzpicture}
				%\draw[help lines] (-6,6) grid (6,-6);
				\draw (-5,5) rectangle (5,-6);
				\draw[draw = black] (-1.5,1) circle (3);
				\draw[draw = black] (1.5,1) circle (3);
				\draw[draw = black] (0,-2) circle (3);
				\node at (-3,4.5) {\LARGE\textbf{Young}};
				\node at (3,4.5) {\LARGE\textbf{Married}};
				\node at (-3,-4.5) {\LARGE\textbf{Male}};
				\node at (-3,1) {$P(Y \cap R^\prime \cap M^\prime)$};
				\node at (0,2) {$P(Y \cap R)$};
				\node at (-1.8,-1) {$P(Y \cap M)$};
				\node at (0,0) {$P(Y \cap R \cap M)$};
			\end{tikzpicture} \par
			According to the constructed venn diagram, we want $P(Y \cap R^\prime \cap M^\prime)$.
		\end{figure}

		\begin{align*}
			P(Y \cap R \cap M) &= 600 \\
			P(Y \cap R) &= 1400 - 600 = 800 \\
			P(Y \cap M) &= 1320 - 600 = 720 \\
			P(Y \cap R^\prime \cap M^\prime) &= 3000 - (600 + 800 + 720) = \boxed{880}
		\end{align*}

	\clearpage

	\section[Chapter 3: Elements of Probability]{Chapter 3 \\
		Elements of Probability}

		\subsection*{3-46}

			\begin{figure}[!h]
				\centering
				\begin{tikzpicture}
					%\draw[help lines] (-4,-3) grid (4,3);
					\draw (-4,-3) rectangle (4,3);
					\draw[draw = black] (-1,0) circle (2);
					\draw[draw = black] (1,0) circle (2);
					\node at (-2,2.5) {\Large\textbf{Lab Work}};
					\node at (2,2.5) {\Large\textbf{Specialist}};
					\node at (0,0) {$P(L \cap S)$};
					\node at (1.5,-2.5) {$P(L^\prime \cap S^\prime)$};
					\node at (-2,0) {$P(L \cap S^\prime)$};
					\node at (2,0) {$P(L^\prime \cap S)$};
				\end{tikzpicture} \par
				According to the constructed venn diagram, we want $P(L \cap S)$.
			\end{figure}
			\[
				P(L^\prime \cap S^\prime) = 0.35 = P(L \cup S)^\prime
			\]
			\[
				P(L) = 0.40 \quad P(S) = 0.30
			\]
			\[
				P(L^\prime) = 0.60 \quad P(S^\prime) = 0.70
			\]
			\begin{align*}
				P(L \cap S) &= P(L) + P(S) - P(L \cup S) \\
							&= \left[ 1 - P(L^\prime) \right] + \left[ 1 - P(S^\prime) \right] - \left[ 1 - P(L \cup S)^\prime \right] \\
							&= \left[ 1 - 0.60 \right] + \left[ 1 - 0.70 \right] - \left[ 1 - 0.35 \right] \\
							&= \boxed{0.05}
			\end{align*}

	\qline

		\subsection*{3-47}

			\begin{equation*}
				\begin{split}
					P(A \cup B) = 0.7 &= P(A) + P(B) - P(A \cap B) \\
					P(A \cup B^{\prime}) = 0.9 &= P(A) + P(B^{\prime}) - P(A \cap B^{\prime}) \\\\
					1.6 &= 2P(A) + P(B) + P(B^\prime) - P(A \cap B) - P(A \cap B^\prime) \\
					1.6 &= 2P(A) + P(B) + \left[ 1 - P(B) \right] - P(A \cap B) - \left[ P(A) - P(A \cap B) \right] \\
					1.6 &= P(A) + 1 \\
					P(A) &= \boxed{0.6}
				\end{split}
			\end{equation*}

	\qline

		\subsection*{3-48}

			\[
				P(>1) = 0.64 \quad P(S) = 0.2
			\]
			\[
				P(1) = 0.36 \quad P(S^\prime) = 0.8
			\]
			\begin{align*}
				P(S^\prime) &= P(>1 \cap S^\prime) + P(1 \cap S^\prime) \\
				P(1 \cap S^\prime) &= P(S^\prime) - P(>1 \cap S^\prime) \\
				&= 0.8 - (0.64 \times 0.85) \\
				&= \boxed{0.256}
			\end{align*}

	\clearpage

		\subsection*{3-49}

		\begin{figure}[!h]
			\centering
			\begin{tikzpicture}
				%\draw[help lines] (-4,-3) grid (4,3);
				\draw (-4,-3) rectangle (4,3);
				\draw[fill = blue, fill opacity=0.4, draw = black] (-1,0) circle (2);
				\draw[draw = black] (1,0) circle (2);
				\node at (-2,2.5) {\Large\textbf{Physical}};
				\node at (2,2.5) {\Large\textbf{Chiropractor}};
				\node at (0,0) {$P(P \cap C)$};
				\node at (1.5,-2.5) {$P(P \cup C)^\prime$};
				\node at (-2,0) {$P(P \cap C^\prime)$};
				\node at (2,0) {$P(P^\prime \cap C)$};
			\end{tikzpicture} \par
			According to the constructed venn diagram, we want the blue shaded area $P(P)$.
		\end{figure}
		\[
			P(P \cap C) = 0.22 \quad P(P \cup C)^\prime	= 0.12
		\]
		\[
			P(C) = P(P) + 0.14
		\]
		\begin{align*}
			P(P \cap C) &= P(P) + P(C) - P(P \cup C) \\
			0.22 &= P(P) + \left[ P(P) + 0.14 \right] - \left[ 1 - 0.12 \right] \\
			P(P) &= \boxed{0.48}
		\end{align*}

	\clearpage

	\subsection*{3-50}

		\begin{figure}[!h]
			\centering
			\begin{tikzpicture}
				%\draw[help lines] (-6,6) grid (6,-6);
				\draw (-5,5) rectangle (5,-6);
				\draw[draw = black] (-1.5,1) circle (3);
				\draw[draw = black] (1.5,1) circle (3);
				\draw[draw = black] (0,-2) circle (3);
				\node at (-3,4.5) {\LARGE\textbf{Gymnastics}};
				\node at (3,4.5) {\LARGE\textbf{Baseball}};
				\node at (-3,-4.5) {\LARGE\textbf{Soccer}};
				\node at (-3,1) {$P(G \cap B^\prime \cap S^\prime)$};
				\node at (3,1) {$P(G^\prime \cap B \cap S^\prime)$};
				\node at (0,-3) {$P(G^\prime \cap B^\prime \cap S)$};
				\node at (3.2,-4.5) {$P(G^\prime \cap B^\prime \cap S^\prime)$};
				\node at (0,2) {$P(G \cap B \cap S^\prime)$};
				\node at (-1.8,-1.2) {\footnotesize$P(G \cap B^\prime \cap S)$};
				\node at (1.8,-1.2) {\footnotesize$P(G^\prime \cap B \cap S)$};
				\node at (0,0) {$P(G \cap B \cap S)$};
			\end{tikzpicture} \par
			According to the constructed venn diagram, we want $P(G^\prime \cap B^\prime \cap S^\prime)$.
		\end{figure}

		\begin{align*}
			P(G \cap B \cap S) &= 0.08 \\\\
			P(G \cap B^\prime \cap S) &= 0.10 - 0.08 = 0.02 \\
			P(G^\prime \cap B \cap S) &= 0.12 - 0.08 = 0.04 \\
			P(G \cap B \cap S^\prime) &= 0.14 - 0.08 = 0.06	\\\\
			P(G \cap B^\prime \cap S^\prime) &= 0.28 - 0.06 - 0.02 - 0.08 = 0.12 \\
			P(G^\prime \cap B \cap S^\prime) &= 0.29 - 0.06 - 0.04 - 0.08 = 0.11 \\
			P(G^\prime \cap B^\prime \cap S) &= 0.19 - 0.02 - 0.04 - 0.08 = 0.05 \\\\
			P(G^\prime \cap B^\prime \cap S^\prime) &= 1 - (0.05 + 0.11 + 0.12 + 0.06 + 0.04 + 0.02) = \boxed{0.52}
		\end{align*}

	\clearpage

	\subsection*{3-51}

		The events are independent and that $P(C)=2P(D)$.
		\begin{align*}
			P(C \cap D) = P(C) \cdot P(D) &= 0.15 \\
			2P(D) \cdot P(D) &= 0.15 \\
			P(D) &= \sqrt{\frac{0.15}{2}} = 0.274 \\
			P(C) &= 2 \cdot 0.274 = 0.548
		\end{align*}
		\begin{align*}
			P(C \cup D) &= P(C) + P(D) - P(C \cap D) \\
			1 - P(C \cup D)^\prime &= 0.548 + 0.274 - 0.15 \\
			P(C \cup D)^\prime &= \boxed{0.328}
		\end{align*}

	\qline

	\subsection*{3-52}

		The events are independent.
		\[
			P(E \cup O) = 0.85
		\]
		\[
			P(E^\prime) = 0.25 \quad P(E) = 0.75
		\]
		\begin{align*}
			P(E \cup O) &= P(E) + P(O) - P(E \cap O) \\
			0.85 &= 0.75 + P(O) - 0.75 \cdot P(O) \\
			P(O) &= \boxed{0.4}
		\end{align*}

	\qline

	\subsection*{3-53}

		\begin{center}
			\renewcommand\arraystretch{1.2}
			\begin{tabular}{c|c|c|c|c|c}
				N & $0$ & $1$ & $2$ & $3$ & $4$ \\ \hline
				p(n) & $\frac{1}{2}$ & $\frac{1}{6}$ & $\frac{1}{12}$ & $\frac{1}{20}$ & $\frac{1}{30}$ \\
			\end{tabular}
		\end{center}

		\begin{align*}
			P(N \geq 1 | N \leq 4) &= \frac{P(N \geq 1 \cap N \leq 4)}{P(N \leq 4)}
			= \dfrac{\frac{1}{6} + \frac{1}{12} + \frac{1}{20} + \frac{1}{30}}{\frac{1}{2} + \frac{1}{6} + \frac{1}{12} + \frac{1}{20} + \frac{1}{30}}
			= \boxed{\frac{2}{5}}
		\end{align*}

	\qline

	\subsection*{3-54}

		We want $ P(H | P^\prime) = \dfrac{P(H \cap P^\prime)}{P(P^\prime)} $
		\[ P(H) = \frac{210}{937} = P(H \cap P) + P(H \cap P^\prime) \]
		\[ P(P) = \frac{312}{937} \quad P(H \cap P) = \frac{102}{937} \]
		\[ P(H \cap P^\prime) = \dfrac{ \dfrac{210}{937} - \dfrac{102}{937} }{\dfrac{625}{937}} = \frac{108}{625} = \boxed{0.1728} \]

	\clearpage

	\subsection*{3-55}

		\begin{figure}[!h]
			\centering
			\begin{tikzpicture}
				%\draw[help lines] (-4,-4) grid (4,4);
				\draw[draw = black] (2.5,0) circle (2);
				\draw[draw = black] (-2.5,0) circle (2);
				\node at (-2.5,2.5) {\LARGE \textbf{Urn A}};
				\node[align=center] at (-2.5,0) {\large 4 R \\ \large 16 B};
				\node at (2.5,2.5) {\LARGE \textbf{Urn B}};
				\node[align=center] at (2.5,0) {\large 16 R \\ \large $x$ B};
			\end{tikzpicture}
		\end{figure}

	\clearpage

	\subsection*{3-56}

		\begin{figure}[!h]
			\centering
			\begin{tikzpicture}
				%\draw[help lines] (-6,6) grid (6,-6);
				\draw (-5,5) rectangle (5,-6);
				\draw[draw = black] (-1.5,1) circle (3);
				\draw[draw = black] (1.5,1) circle (3);
				\draw[draw = black] (0,-2) circle (3);
				\node at (-3,4.5) {\LARGE\textbf{A}};
				\node at (3,4.5) {\LARGE\textbf{B}};
				\node at (-3,-4.5) {\LARGE\textbf{C}};
				\node at (-3,1) {\LARGE$0.1$};
				\node at (3,1) {\LARGE$0.1$};
				\node at (0,-3) {\LARGE$0.1$};
				\node at (3.2,-5) {$P(A^\prime \cap B^\prime \cap C^\prime)$};
				\node at (0,2) {\LARGE$0.12$};
				\node at (-1.8,-1) {\LARGE$0.12$};
				\node at (1.8,-1) {\LARGE$0.12$};
				\node at (0,0) {$P(A \cap B \cap C)$};
			\end{tikzpicture} \par
		\end{figure}

		We want $ P((A^\prime \cap B^\prime \cap C^\prime) | A^\prime) = \dfrac{P((A^\prime \cap B^\prime \cap C^\prime) \cap A^\prime)}{P(A^\prime)} = \dfrac{P(A^\prime \cap B^\prime \cap C^\prime)}{P(A^\prime)} $
		\[ 0.10 = P(A \cap B^\prime \cap C^\prime) = P(A^\prime \cap B \cap C^\prime) = P(A^\prime \cap B^\prime \cap C)  \]
		\[ 0.12 = P(A \cap B \cap C^\prime) = P(A \cap B^\prime \cap C) = P(A^\prime \cap B \cap C)  \]
		\[ \frac{1}{3} = P(A \cap B \cap C | A \cap B) = \frac{P(A \cap B \cap C \cap A \cap B)}{P(A \cap B)} = \frac{P(A \cap B \cap C)}{P(A \cap B)} = \frac{P(A \cap B \cap C)}{P(A \cap B \cap C) + 0.12} \]
		\begin{align*}
			P(A \cap B \cap C | A \cap B) = \frac{P(A \cap B \cap C \cap A \cap B)}{P(A \cap B)} = \frac{P(A \cap B \cap C)}{P(A \cap B)} &= \frac{1}{3} \\
			\frac{P(A \cap B \cap C)}{P(A \cap B \cap C) + 0.12} &= \frac{1}{3} \\
			P(A \cap B \cap C) &= 0.06
		\end{align*}
		\begin{align*}
			\frac{P(A^\prime \cap B^\prime \cap C^\prime)}{P(A^\prime)} = \frac{1 - (3(0.1) + 3(0.12) + 0.06)}{1 - (0.1 + 2(0.12) + 0.06)} = \frac{0.28}{0.6} = \boxed{0.467}
		\end{align*}

	\clearpage

		\subsection*{3-57}

			\begin{figure}[!h]
				\centering
				\begin{tikzpicture}
					%\draw[help lines] (-6,6) grid (6,-6);
					\draw (-5,5) rectangle (5,-6);
					\draw[draw = black] (-1.5,1) circle (3);
					\draw[draw = black] (1.5,1) circle (3);
					\draw[draw = black] (0,-2) circle (3);
					\node at (-3,4.5) {\LARGE\textbf{A}};
					\node at (3,4.5) {\LARGE\textbf{B}};
					\node at (-3,-4.5) {\LARGE\textbf{C}};
					\node at (-3,1) {\LARGE$0$};
					\node at (3,1) {\LARGE$0$};
					\node at (0,-3) {\LARGE$0$};
					\node at (3.2,-5) {$P(A^\prime \cap B^\prime \cap C^\prime)$};
					\node at (0,2) {$P(A \cap B \cap C^\prime)$};
					\node at (-1.8,-1.2) {\footnotesize$P(A \cap B^\prime \cap C)$};
					\node at (1.8,-1.2) {\footnotesize$P(A^\prime \cap B \cap C)$};
					\node at (0,0) {\LARGE$0$};
				\end{tikzpicture} \par
			\end{figure}

			\begin{align*}
				P(A) &= \frac{1}{4} = P(A \cap B \cap C^\prime) + P(A \cap B^\prime \cap C) \\
				P(B) &= \frac{1}{3} = P(A^\prime \cap B \cap C) + P(A \cap B \cap C^\prime) \\
				P(C) &= \frac{5}{12} = P(A^\prime \cap B \cap C) + P(A \cap B^\prime \cap C) \\
				P(A^\prime \cap B^\prime \cap C^\prime) &= 1 - \left[ P(A \cap B \cap C^\prime) + P(A \cap B^\prime \cap C) + P(A^\prime \cap B \cap C) \right] \\
 			\end{align*}



	\section[Chapter 4: Discrete Random Variables]{Chapter 4 \\
		Discrete Random Variables}

		\subsection*{4-18}

	\clearpage

	\section[Chapter 5: Commonly Used Discrete Distributions]{Chapter 5 \\
		Commonly Used Discrete Distributions}

		\subsection*{5-36}

	\clearpage

	\section[Chapter 6: Applications for Discrete Random Variables]{Chapter 6 \\
		Applications for Discrete Random Variables}
	\vspace{-0.2cm}
	\hrule
	\vspace{0.5cm}
		\subsection*{6-17}

			\begin{center}
				\begin{tabular}{c|c|c|c|c|c}
					x & $0$ & $1$ & $2$ & $3$ & $\cdots$ \\
					\hline
					p(x) &  &  &  &  & $\cdots$ \\
				\end{tabular} $\Rightarrow$
				\begin{tabular}{c|c|c|c|c|c}
					y & $0$ & $1000$ & $2000$ & $2000$ & $\cdots$ \\
					\hline
					p(y) &  &  &  &  & $\cdots$ \\
				\end{tabular}
				\\
				\begin{tabular}{c|c|c|c|c|c}
					2000x & $0$ & $2000$ & $2000$ & $2000$ & $\cdots$ \\
					\hline
					p(x) &  &  &  &  & $\cdots$ \\
				\end{tabular}
			\end{center}

	\clearpage

	\section[Chapter 7: Continuous Random Variables]{Chapter 7 \\
		Continuous Random Variables}

		\subsection*{7-12}

			\begin{equation*}
				\begin{split}
					f(x) &\propto \left( 10 + x \right)^{-2} \\
					f(x) &= K \left( 10 + x \right)^{-2} \quad ; \quad 0 \leq x \leq 40
				\end{split}
			\end{equation*}

			\begin{equation*}
				\begin{split}
					\int_{0}^{40} K \left( 10 + x \right)^{-2} &= 1 \\
					1 &= K \left[ \frac{(10 + x)^{-1}}{-1} \right]^{40}_0 \\
					\frac{1}{K} &= \left[ \left(-\frac{1}{50}\right) - \left(-\frac{1}{10}\right) \right] = \frac{2}{25} \\
					K &= \frac{25}{2}
				\end{split}
			\end{equation*}

			\[
				f(x) = \frac{25}{2} \left( 10 + x \right)^{-2} \quad ; \quad 0 \leq x \leq 40
			\]

			\begin{equation*}
				\begin{split}
					P(X < 6) &= \int_{0}^{6} \frac{25}{2} \left( 10 + x \right)^{-2} \\
					&= \frac{25}{2} \left[ -(10 + x)^{-1} \right]^{6}_0 \\
					&= \frac{25}{2} \left[ \left(-\frac{1}{16}\right) - \left(-\frac{1}{10}\right) \right] \\
					&= \frac{15}{32} = \boxed{\num{0.46875}}
				\end{split}
			\end{equation*}

		\subsection*{7-13}

			\begin{equation*}
				\begin{split}
					k^{th} \text{ Percentile : } \int_{200}^{P_{k}} \frac{2.5 (200)^{2.5}}{x^{3.5}} \ dx &= \frac{k}{100} \\
					\frac{k}{100} &= (2.5 (200)^{2.5}) \int_{200}^{P_{k}} x^{-3.5} \\
					\frac{k}{100(2.5 (200)^{2.5})} &= \left[ \frac{x^{-2.5}}{-2.5} \right]^{P_{k}}_{200} \\
					\frac{(k)(-2.5)}{100(2.5 (200)^{2.5})} &= \left[ P_{k}^{-2.5} - 200^{-2.5} \right] \\
					P_{k}^{-2.5} &= 200^{-2.5} - \frac{k}{100(200)^{2.5}} = \frac{100(200)^{2.5}(200)^{-2.5} - k}{100(200)^{2.5}} \\
					P_{k}^{2.5} &= \frac{100(200)^{2.5}}{100 - k} \\
					P_{k} &= \left[\frac{100(200)^{2.5}}{100 - k}\right]^{1/2.5} \\
				\end{split}
			\end{equation*}

			\[
				P_{70} - P_{30} = 323.73 - 230.670 = \boxed{93.06}
			\]

	\clearpage

		\subsection*{7-14}

			\begin{equation*}
				\begin{split}
					f(x) &\propto \left( 1 + x \right)^{-4} \\
					f(x) &= K \left( 1 + x \right)^{-4} \quad ; \quad 0 \leq x \leq \infty
				\end{split}
			\end{equation*}

			\begin{equation*}
				\begin{split}
					\int_{0}^{\infty} K \left( 1 + x \right)^{-4} &= 1 \\
					1 &= K \left[ \frac{(1 + x)^{-3}}{-3} \right]^{\infty}_0 \\
					\frac{1}{K} &= \left[ 0 -  \left(-\frac{1}{3}\right) \right] = \frac{1}{3} \\
					K &= 3
				\end{split}
			\end{equation*}

			\[
				f(x) = 3 \left( 1 + x \right)^{-4} \quad ; \quad 0 \leq x \leq \infty
			\]
			\begin{equation*}
				\begin{split}
					E\left[X\right] &= \int_{0}^{\infty} 3x\left( 1 + x \right)^{-4} \\
					&= 3 \left[ \frac{(1 + x)^{-3} (1-3x)}{6} \right]^{\infty}_0 = \left[ 0 - \left( \frac{1}{2} \right) \right] = \boxed{\frac{1}{2}}
				\end{split}
			\end{equation*}

	\qline

		\subsection*{7-15}

			\[
				f(x) = \frac{\abs{x}}{10} \quad \text{for } -2 \leq x \leq 4
				\begin{cases}
					\dfrac{-x}{10}  &  \text{for } -2 \leq x \leq 0 \\
					\\
					\dfrac{x}{10}   &  \text{for } 0  \leq x \leq 4
				\end{cases}
			\]

			\begin{equation*}
				\begin{split}
					E\left[X\right] &= \int_{-2}^{0} \frac{-x^2}{10} \ dx + \int_{0}^{4} \frac{x^2}{10} \ dx \\
					&= \left[ \frac{-x^3}{30} \right]^0_{-2} + \left[ \frac{x^3}{30} \right]^4_{0} = \left[ \left( \frac{0}{30} \right) - \left( \frac{8}{30} \right) \right] + \left[ \left( \frac{64}{30} \right) - \left( \frac{0}{30} \right) \right] = \boxed{\frac{28}{15}}
				\end{split}
			\end{equation*}

	\qline

		\subsection*{7-16}

			\[
				P(X > 16 \ | \ X > 8) = \frac{P(X > 16)}{P(X > 8)} = \frac{1 - P(X < 16)}{1 - P(X < 8)} = \frac{1 - \int_{0}^{16} 0.005(20 - x) \ dx}{1 - \int_{0}^{8} 0.005(20 - x) \ dx}
			\]

			\begin{equation*}
				\begin{split}
					P(X < x) &= \int_{0}^{x} 0.005(20 - x) \ dx = 0.005 \left[ \int_{0}^{x} 20 \ dx - \int_{0}^{x} x \ dx \right] \\
					&= 0.005 \left[ \left[ 20x \right]^x_0 - \left[ \frac{x^2}{2} \right]^x_0 \right] = 0.1x - 0.0025x^2 \\
				\end{split}
			\end{equation*}

			\[
				P(X > 16 \ | \ X > 8) = \frac{1 - 0.96}{1 - 0.64} = \boxed{\frac{1}{9}}
			\]

	\qline

		\subsection*{7-17}

			\[
				P(X < 2 \ | \ X \geq 1.5) = \frac{P(1.5 \leq X \leq 2)}{P(X \geq 1.5)} = \frac{P(1.5 < X < 2)}{1 - P(X < 1.5)} = \frac{\int_{1.5}^{2} 3x^{-4} \ dx}{1 - \int_{1}^{1.5} 3x^{-4} \ dx}
			\]

			\begin{equation*}
				\begin{split}
					P(a \leq X \leq b) = \int_{a}^{b} 3x^{-4} \ dx = \left[ -x^{-3} \right]^a_b = (b)^{-3} - (a)^{-3}
				\end{split}
			\end{equation*}

			\[
				P(X < 2 \ | \ X \geq 1.5) = \frac{\left( \dfrac{37}{216} \right)}{1 - \left( \dfrac{19}{27} \right)} = \frac{37}{64} = \boxed{0.578125}
			\]

	\clearpage

	\section[Chapter 8: Commonly Used Continuous Distributions]{Chapter 8 \\
		Commonly Used Continuous Distributions}

		\subsection*{8-56}

			\begin{equation*}
				\begin{split}
					F(4) = 1 - e^{-4\lambda} &= 0.5 \\
					\lambda &= -\frac{\ln(0.5)}{4}
				\end{split}
			\end{equation*}

			\begin{equation*}
				\begin{split}
					P(X \geq 5) &= 1 - P(X \leq 5) \\
					&= 1 - \left( 1 - e^{-5\left( -\frac{\ln(0.5)}{4} \right)} \right) \\
					&= \boxed{\num{0.42045}}
				\end{split}
			\end{equation*}

	\qline

		\subsection*{8-57}

			\begin{equation*}
				\begin{split}
					E[G] &= 6 = \frac{1}{\lambda_\mathrm{G}} \Rightarrow \lambda_\mathrm{G} = \frac{1}{6} \\
					E[B] &= 3 = \frac{1}{\lambda_\mathrm{B}} \Rightarrow \lambda_\mathrm{B} = \frac{1}{3}
				\end{split}
			\end{equation*}

			\begin{equation*}
				\begin{split}
					P(G \leq 3 \cap V \leq 2) &= P(G \leq 3) \cdot P(V \leq 2) \\
					&= F_\mathrm{G}(3) \cdot F_\mathrm{B}(2) \\
					&= \left( 1 - e^{-\frac{3}{6}} \right) \cdot \left( 1 - e^{-\frac{2}{3}} \right) \\
					&= \boxed{\num{0.19146}}
				\end{split}
			\end{equation*}

	\qline

		\subsection*{8-58}

			\[
				E[G] = 2 = \frac{1}{\lambda} \Rightarrow \lambda = \frac{1}{2}
			\]
			\[
				F(x) = 1 - e^{-\frac{1}{2} x}
			\]
			\begin{center}
				\textbf{Probability model for each printer} \\
				\vspace{5mm}
				\begin{tabular}{c|c|c}
					$x$ & $1$ & $2$ \\
					\hline
					p(x) & $F(1)$ & $F(2)-F(1)$ \\
				\end{tabular}
				$\quad \Rightarrow \quad$
				\begin{tabular}{c|c|c}
					y & $200$ & $100$ \\
					\hline
					p(y) & $F(1)$ & $F(2)-F(1)$ \\
				\end{tabular}
			\end{center}
			\par
			\begin{equation*}
				\begin{split}
					E[Y] &= \sum Y \cdot P(Y) \\
					&= 200 \left( 1 - e^{-\frac{1}{2}} \right) + 100 \left[ \left( 1 - e^{-1} \right) - \left( 1 - e^{-\frac{1}{2}} \right)\right] \\
					&= 102.56 \\
					100 \cdot E[Y] &= \boxed{\num{10256}}
				\end{split}
			\end{equation*}

	\clearpage

		\subsection*{8-59}

			\[
				P(X \leq 50) = 0.3
			\]
			\begin{equation*}
				\begin{split}
					F(50) = 1 - e^{\frac{-50}{\lambda}} &= 0.3 \\
					\lambda &= -\frac{50}{\ln(0.7)}
				\end{split}
			\end{equation*}
			\begin{equation*}
				\begin{split}
					P(X \leq 80) = F(80) &= 1 - e^{\frac{-80}{\lambda}} \\
					&= 1 - e^{\frac{-80}{\left(-\frac{50}{\ln(0.7)}\right)}} \\
					&= \boxed{\num{0.43486}}
				\end{split}
			\end{equation*}

	\qline

		\subsection*{8-60}

			\[
				f(x) = c e^{-0.004x} \text{ for } x \geq 0
			\]
			\[
				f(x) = \lambda e^{-\lambda x} \text{ for } x \geq 0
			\]
			\[
				c = 0.004
			\]
			\begin{equation*}
				\begin{split}
					f(x) &= 0.004e^{-0.004x} \\
					F(x) &= 1 - e^{-0.004x}
				\end{split}
			\end{equation*}
			\begin{equation*}
				\begin{split}
					F(M) = 1 - e^{-0.004M} &= 0.5 \\
					M &= \frac{\ln(0.5)}{-0.004} \\
					&= \boxed{\num{173.29}}
				\end{split}
			\end{equation*}

	\qline

		\subsection*{8-61}

			\begin{center}
				\renewcommand\arraystretch{1.2}
				\begin{tabular}{c|c|c|c}
					N & $0$ & $1$ & $>1$ \\ \hline
					p(n) & $\frac{1}{2}$ & $\frac{1}{3}$ & $\frac{1}{6}$ \\
				\end{tabular} \vspace{5mm}
				\par
				Three different probability densities for each $N$. $P(4 < S < 8)$ means at least 1 claim was insured for a given year, therefore $P(N=0)$ is not included.
			\end{center}

			\begin{equation*}
				\begin{split}
					F_{\exp(5)}(x) &= 1 - e^{-\frac{x}{5}} \\
					F_{\exp(8)}(x) &= 1 - e^{-\frac{x}{8}} \\
				\end{split}
			\end{equation*}

			\begin{equation*}
				\begin{split}
					P(4 < S < 8) &= \frac{1}{3} \left[ F_{\exp(5)}(8) - F_{\exp(5)}(4) \right] + \frac{1}{6} \left[ F_{\exp(8)}(8) - F_{\exp(8)}(4) \right] \\
					&= \frac{1}{3} \left[ \left( 1 - e^{-\frac{8}{5}} \right) - \left( 1 - e^{-\frac{4}{5}} \right) \right] + \frac{1}{6} \left[ \left( 1 - e^{-\frac{8}{8}} \right) - \left( 1 - e^{-\frac{4}{8}} \right) \right] \\
					&= \boxed{\num{0.12225}}
				\end{split}
			\end{equation*}

	\clearpage

		\subsection*{8-62}

			\[
				X \sim Po(2) \quad \lambda = 2 = \sigma^2
			\]
			\[
				n\lambda = 1250 \cdot 2 = 2500 = \sigma^2 \quad \Rightarrow \quad \sigma = 50
			\]
			\[
				P(2450 \leq X \leq 2600)
			\]
			\begin{center}
				Since independence is assumed, the Central Limit Theorem can be used.
			\end{center}
			\[
				P(z_1 \leq z \leq z_2)
			\]
			\[
				z_1 = \frac{2450 - 2500}{50} = -1 \quad,\quad z_2 = \frac{2600 - 2500}{50} = 2
			\]
			\[
				P(-1 \leq z \leq 2) = 0.9772 - (1 - 0.8413) = \boxed{0.8185}
			\]

	\qline

		\subsection*{8-63}

			\[
				f(x) = \frac{1}{1000} e^{-\dfrac{x}{1000}} \quad \Rightarrow \quad \lambda = 1000
			\]

			\[
				P(x) = E[X] + 100 = 1100
			\]
			\begin{equation*}
				\begin{split}
					100P(x) &= 100 \cdot 1100 = 110000 \\
					100E[X] &= 100 \cdot 1000 = \num{e5} \\
					100Var(X) &= 100 \cdot 1000^2 = \num{e8} \\
					\sigma &= \sqrt{\num{e8}} = \num{e4}
				\end{split}
			\end{equation*}

			\[
				X \sim N(\num{e5}, \num{e4}) \quad P(X > 110000)
			\]
			\[
				z = \frac{110000 - 100000}{10000} = 1
			\]
			\[
				P(z > 1) = 1 - P(z < 1) = 1 - 0.8413 = \boxed{\num{0.1587}}
			\]

	\qline

		\subsection*{8-64}

			\begin{center}
				\begin{tabular}{c|c|c|c}
					x & $0$ & $1$ & $2$ \\ \hline
					p(x) & $0.6$ & $0.1$ & $0.3$ \\
				\end{tabular}
				\begin{equation*}
					\begin{split}
						E[X] &= 0.7 = \mu \\
						E[X^2] &= 1.3 \\
						Var(X) &= 0.81 \\
						\sigma &= 0.9
					\end{split}
				\end{equation*}
			\end{center}

			\begin{equation*}
				\begin{split}
					100 \cdot \mu &= 70 \\
					100 \cdot Var(X) &= 81 \\
					\sigma &= 9
				\end{split}
			\end{equation*}

			\[
				X \sim N(70, 9) \quad P(X \leq 90)
			\]
			\begin{center}
				Since we approximate a discrete random variable with a continuous one, an adjustment is needed by continuity correction, which includes the entire block of probability for that value.
			\end{center}
			\[
				P(X \leq 90) \Rightarrow  P(X \leq 90.5)
			\]
			\[
				z = \frac{90.5 - 70}{9} = 2.28
			\]
			\[
				P(z \leq 2.28) = \boxed{0.9887}
			\]

	\clearpage

		\subsection*{8-65}

			\[
				X \sim U(-2.5,2.5) \quad,\quad \mu = 0 \quad,\quad \sigma^2 = \frac{25}{12}
			\]

			\begin{center}
				Because the sample size $n=48$ is large, the sample mean $\bar{x}$ is approximately normally distributed, with mean $\mu$ and standard deviation $\frac{\sigma}{\sqrt{n}}$.
			\end{center}

			\begin{equation*}
				\begin{split}
					 P(-0.25 \leq \bar{x} \leq 0.25) &= P\left( \frac{-0.25 - \mu}{\frac{\sigma}{\sqrt{n}}} \leq z \leq \frac{0.25 - \mu}{\frac{\sigma}{\sqrt{n}}} \right) \\
					 &= P(-1.2 \leq z \leq 1.2) \\
					 &= 0.8849 - (1 - 0.8849) \\
					 &= \boxed{\num{0.7698}}
				\end{split}
			\end{equation*}

	\qline

		\subsection*{8-66}

			\begin{center}
				For individual contributions:
				\[
					\mu = 3125 \quad,\quad \sigma = 250 \quad,\quad Var(x) = \sigma^2 = 62500
				\]
				For $n = 2025$:
				\[
					n\mu = 6328125 \quad,\quad nVar(x) = 126562500 \quad,\quad \sigma = 11250
				\]
			\end{center}

			\[
				X \sim N(6328125, 11250) \quad,\quad P(X \leq z) = 0.9
			\]
			\begin{center}
				$0.9$ is found between $z=1.28$ and $z=1.29$, with z-scores $0.8997$ and $0.9015$ respectively. \\
				\[
					(0.01)(\frac{0.0003}{0.9015-0.8997}) = \frac{1}{600}
				\]
			\end{center}
			\par
			\begin{equation*}
				\begin{split}
					z = \frac{x - 6328125}{11250} &= 1.28 + \frac{1}{600} \\
					x &= \boxed{\num{6342543.75}}
				\end{split}
			\end{equation*}


	\clearpage

	\section[Chapter 9: Applications for Continuous Random Variables]{Chapter 9 \\
		Applications for Continuous Random Variables}

		\subsection*{9-26}

			\[
				P(Y < 0.5) = 0.64 \quad;\quad Y = X - C
			\]
			\begin{equation*}
				\begin{split}
					P(X-C < 0.5) &= 0.64 \\
					P(X < 0.5 + C) &= 0.64 \\
					0.64 &= \int_{0}^{0.5 + C} 2x \ dx = \left[ x^2 \right]^{0.5 + C}_0 \\
					0.64 &= (0.5 + C)^2 \\
					C &= \pm 0.8 - 0.5 \quad;\quad 0 < C < 1 \\
					C &= \boxed{0.3}
				\end{split}
			\end{equation*}

	\qline

		\subsection*{9-27}

			\[
				\text{Set } Y = \text{losses paid by the manufacturer with deductible}
			\]
			\begin{equation*}
				Y =
				\begin{cases}
					\ X  &; \quad  0.6 < x \leq 2 \\
					\ 2  &; \quad  x > 2
				\end{cases}
			\end{equation*}
			\begin{equation*}
				\begin{split}
					E[Y] &= \int_{0.6}^{2} x\left( \frac{2.5(0.6)^{2.5}}{x^{3.5}} \right) + \int_{2}^{\infty} 2\left( \frac{2.5(0.6)^{2.5}}{x^{3.5}} \right) \\
					&= 2.5(0.6)^{2.5} \left[ \left( \frac{1}{-1.5(2)^{1.5}} - \frac{1}{-1.5(0.6)^{1.5}} \right) + \left( 0 - \frac{2}{-2.5(2)^{2.5}} \right) \right] \\
					E[Y] &= \boxed{0.93427}
				\end{split}
			\end{equation*}

	\qline

		\subsection*{9-28}

			\[
				X \sim U[0, 1000] \quad,\quad f(x) = \frac{1}{1000 - 0} = \frac{1}{1000} \quad,\quad E[X] = \frac{1000 + 0}{2} = 500
			\]
			\par
			\[
				E[Y] = 0.25 \cdot E[X] = 125
			\]
			\begin{equation*}
				\begin{split}
					E[Y] &= \int_{0}^{d} (0)\left( \frac{1}{1000} \right) \ dx + \int_{d}^{1000} (x - d)\left( \frac{1}{1000} \right) \ dx = 125 \\
					125 &= \left[ \frac{(x-d)^2}{2000} \right]^{1000}_d \\
					125 &= \left[ \left( \frac{(1000-d)^2}{2000} \right) - \left( \frac{(d-d)^2}{2000} \right) \right] \\
					(1000-d)^2 &= 250000 \\
					1000 - d &= \pm 500 \\
					d &= 1000 \pm 500 \quad;\quad 0 < d < 1000 \\
					d &= \boxed{500}
				\end{split}
			\end{equation*}

	\clearpage

		\subsection*{9-29}

			\[
				X \sim \exp(10) \quad;\quad f(x) = \frac{1}{10} \exp(-\frac{x}{10})
			\]

			\begin{center}
				\begin{tabular}{c|c|c|c|c|c}
					Y & $1$ & $2$ & $3$ & $4$ & $\cdots$ \\ \hline
					p(y) & $x$ & $\frac{x}{2}$ & $\frac{x}{2}$ & $0$ & $\cdots$ \\
				\end{tabular}
			\end{center}

			\begin{equation*}
				\begin{split}
					E[Y] &= \int_{0}^{1} x \cdot \frac{1}{10} \exp(-\frac{t}{10}) \ dt + \int_{1}^{3} \frac{x}{2} \cdot \frac{1}{10} \exp(-\frac{t}{10}) + \int_{3}^{\infty} 0 \cdot \frac{1}{10} \exp(-\frac{t}{10}) \ dt \\
					1000 &= x \left[ -\exp\left( -\frac{t}{10} \right) \right]^1_0 + x \left[ -\frac{1}{2} \exp\left( -\frac{t}{10} \right) \right]^3_1 \\
					1000 &= x \left[ \left( -\exp\left( -\frac{1}{10} \right) + 1 \right) + \left( -\frac{1}{2} \exp\left( -\frac{3}{10} \right) + \frac{1}{2} \exp\left( -\frac{1}{10} \right) \right) \right] \approx 0.17717x \\
					x &= \boxed{\num{5664.3}}
				\end{split}
			\end{equation*}

	\qline

		\subsection*{9-30}

			\[
				X \sim \exp(3) \quad;\quad f(x) = \frac{1}{3} \exp\left( -\frac{x}{3} \right) \quad;\quad F(x) = 1 - \exp\left( -\frac{x}{3} \right)
			\]

			\begin{equation*}
				\begin{split}
					E[X] &= \int_{0}^{2} 2 \cdot \frac{1}{3} \exp\left( -\frac{x}{3} \right) \ dx + \int_{2}^{\infty} x \cdot \frac{1}{3} \exp\left( -\frac{x}{3} \right) \ dx \\
					&= 2\left[ 1 - \exp\left( -\frac{2}{3} \right) \right] + \frac{1}{3} \left[ -3x\exp\left( -\frac{x}{3} \right) - 9\exp\left( -\frac{x}{3} \right) \right]^{\infty}_2 \\
					&= \left[ 2 - 2\exp\left( -\frac{2}{3} \right) \right] + \left[ 2\exp\left( -\frac{2}{3} \right) + 3\exp\left( -\frac{2}{3} \right) \right] \\
					E[X] &= 2 + 3\exp\left( -\frac{2}{3} \right) \approx \boxed{\num{3.5403}} \\
				\end{split}
			\end{equation*}

	\qline

		\subsection*{9-31}

			\begin{equation*}
				\begin{split}
					E[Y] &= \int_{1}^{10} y \cdot 2y^{-3} \ dy + \int_{10}^{\infty} 10 \cdot 2y^{-3} \ dy \\
					&= \left[ -\frac{2}{y} \right]^{10}_1 + \left[ -\frac{10}{y^2} \right]^{\infty}_{10} \\
					&= \left( -\frac{2}{10} + \frac{2}{1} \right) + \left( 0 + \frac{10}{100} \right) \\
					E[Y] &= \boxed{\num{1.9}}
				\end{split}
			\end{equation*}

	\clearpage

		\subsection*{9-32}

			\begin{equation*}
				\begin{split}
					E[Y] &= \int_{0}^{4} x \cdot \frac{1}{5} \ dx + \int_{4}^{5} 4 \cdot \frac{1}{5} \ dx \\
					&= \left[ \frac{x^2}{10} \right]^4_0 + \left[ \frac{4x}{5} \right]^5_4 \\
					E[Y] &= \frac{12}{5} \\
					E[Y^2] &= \int_{0}^{4} x^2 \cdot \frac{1}{5} \ dx + \int_{4}^{5} 4^2 \cdot \frac{1}{5} \ dx \\
					&= \left[ \frac{x^3}{15} \right]^4_0 + \left[ \frac{16x}{5} \right]^5_4 \\
					E[Y^2] &= \frac{112}{15} \\
					Var(Y) &= E[Y^2] - \left( E[Y] \right)^2 \\
					&= \frac{112}{15} - \left( \frac{12}{5} \right)^2 \\
					Var(Y) &= \frac{128}{75} \approx \boxed{\num{1.7067}}
				\end{split}
			\end{equation*}

	\qline

		\subsection*{9-33}

			\[
				X \sim U(0, 1500) \quad,\quad f(x) = \frac{1}{1500 - 0} = \frac{1}{1500}
			\]

			\begin{equation*}
				\begin{split}
					E[Y] &= \int_{0}^{250} 0 \cdot \frac{1}{1500} \ dx + \int_{250}^{1500} (x-250) \cdot \frac{1}{1500} \ dx \\
					&= \left[ \frac{(x-250)^2}{3000} \right]^{1500}_{250} \\
					E[Y] &= \frac{3125}{6} \\
					E[Y^2] &= \int_{0}^{250} 0^2 \cdot \frac{1}{1500} \ dx + \int_{250}^{1500} (x-250)^2 \cdot \frac{1}{1500} \ dx \\
					&= \left[ \frac{(x-250)^3}{4500} \right]^{1500}_{250} \\
					E[Y^2] &= \frac{3906250}{9} \\
					Var(Y) &= E[Y^2] - \left( E[Y] \right)^2 \\
					&= \frac{3906250}{9} - \left( \frac{3125}{6} \right)^2 \\
					Var(Y) &= \num{162760.4167} = \sigma^2 \\
					\sigma &= \sqrt{\num{162760.4167}} = \boxed{\num{403.436}}
				\end{split}
			\end{equation*}

	\clearpage

		\subsection*{9-34}

			\[
				X \sim \exp(300) \quad,\quad f(x) = \frac{1}{300} \exp\left( -\frac{x}{300} \right) \quad,\quad F(x) = 1 - \exp\left( -\frac{x}{300} \right)
			\]

			\begin{equation*}
				\begin{split}
					P(X > x | X > 100) = 0.95 &= \frac{P(X > x \cap X > 100)}{P(X > 100)} \\
					0.95 &= \frac{\int_{100}^{x} \frac{1}{300} \exp\left( -\frac{x}{300} \right)}{\int_{100}^{\infty} \frac{1}{300} \exp\left( -\frac{x}{300} \right)} = \frac{\left[ -\exp\left( -\frac{x}{300} \right) \right]^x_{100}}{\left[ -\exp\left( -\frac{x}{300} \right) \right]^{\infty}_{100}} = \frac{-\exp\left( -\frac{x}{300} \right) + \exp\left( -\frac{1}{3} \right)}{0 + \exp\left( -\frac{1}{3} \right)} \\
					0.95 &= \frac{-\exp\left( -\frac{x}{300} \right)}{\exp\left( -\frac{1}{3} \right)} + 1 \\
					\exp\left( -\frac{x}{300} \right) &= 0.05 \exp\left( -\frac{1}{3} \right) \\
					x &= -300\ln\left[ 0.05 \exp\left( -\frac{1}{3} \right) \right] \approx \boxed{\num{998.72}}
				\end{split}
			\end{equation*}

	\qline

		\subsection*{9-35}

			\begin{equation*}
				\begin{split}
					G(y) &= P(Y < y) = P(T^2 < y) = P(T < \pm\sqrt{y}) \quad,\quad y > 4 \\
					G(y) &= P(T < \sqrt{y}) = F(\sqrt{y}) = 1 - \left( \frac{2}{\sqrt{y}} \right)^2 \\
					g(y) = G^\prime(y) &= F^\prime(\sqrt{y}) = \frac{d}{dy}\left[ 1 - \frac{4}{y} \right] = \boxed{\frac{4}{y^2}}
				\end{split}
			\end{equation*}

	\qline

		\subsection*{9-36}

			\[
				R \sim \left( 0.04, 0.08 \right) \quad,\quad f(x) = \frac{1}{0.08-0.04} = 25
			\]
			\begin{equation*}
				\begin{split}
					F(v) &= P(V < v) = P(10000\exp(R) < v) = P(R < \ln\left( \frac{v}{10000} \right)) \\
					F(v) &= \int_{0.04}^{\ln\left( \frac{v}{10000} \right)} 25 \ dx = \boxed{25 \left[ \ln\left( \frac{v}{10000} \right) - 0.04 \right]}
				\end{split}
			\end{equation*}

	\qline

		\subsection*{9-37}

			\[
				X \sim \exp(1) \quad,\quad f(x) = \exp\left( -x \right) \quad,\quad F(x) = 1 - \exp\left( -x \right)
			\]
			\begin{equation*}
				\begin{split}
					G(y) &= P(Y < y) = P(10X^{0.8} < y) = P\left( X < \left( \frac{y}{10} \right)^{1.25} \right) = F\left( \left( \frac{y}{10} \right)^{1.25} \right) \\
					G(y) &= F\left( \left( \frac{y}{10} \right)^{1.25} \right) = 1 - \exp\left( -\left( \frac{y}{10} \right)^{1.25} \right) \\
					g(y) = G^\prime(y) &= \frac{d}{dy} \left[ 1 - \exp\left( -\left( \frac{y}{10} \right)^{1.25} \right) \right] = -\left[ -(1.25)\left( \frac{y}{10} \right)^{0.25} \left( \frac{1}{10} \right) \right]\exp\left( - \left( \frac{y}{10} \right)^{1.25} \right) \\
					g(y) &= \boxed{0.125 \exp\left( \left(-0.1y\right)^{1.25} \right)(0.1y)^{0.25}}
				\end{split}
			\end{equation*}

	\clearpage

		\subsection*{9-38}



	\clearpage

	\section[Chapter 10: Multivariate Distributions]{Chapter 10 \\
	Multivariate Distributions}

		\subsection*{10-27}

	\clearpage


\end{document}