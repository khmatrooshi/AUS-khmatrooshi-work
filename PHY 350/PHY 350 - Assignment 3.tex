\documentclass{article}
\input{C:/Users/khali/OneDrive/AUS/Classes/7 - S24/preamble.tex}

%\usepackage{newtx}
\usepackage{microtype}
\usepackage[shortconst]{physconst}
%\usepackage{indentfirst}
\usepackage[nottoc]{tocbibind}

\hypersetup{
	pdftitle={PHY 350 - Assignment 3},
	pdfauthor={Khalifa Salem Almatrooshi},
	%pdfsubject={Your subject here},
	%pdfkeywords={keyword1, keyword2},
	bookmarksnumbered=true,     
	bookmarksopen=true,         
	bookmarksopenlevel=1,       
	colorlinks=true,
	allcolors=blue,
	%linkcolor=blue,
	%filecolor=magenta,      
	%urlcolor=cyan,            
	pdfstartview=Fit,           
	pdfpagemode=UseOutlines,
	pdfpagelayout=TwoPageRight
}

\usepackage{abstract}
\renewcommand{\absnamepos}{flushleft}
\renewcommand{\abstractnamefont}{\large\bfseries}
\renewcommand{\abstracttextfont}{\normalsize}
\setlength{\absleftindent}{0pt}
\setlength{\absrightindent}{0pt}

\setlist[enumerate]{itemsep=12pt, topsep=12pt, partopsep=0pt}

\begin{document}
	
	\begin{center}
		\hrule
		\vspace{0.4cm}
		\textbf { \Large PHY 350 --- Quantum Mechanics}
		\vspace{0.4cm}
	\end{center}
		\bd{Name:} \ Khalifa Salem Almatrooshi \hspace{\fill} \bd{Due Date:} 23 April 2024 \\
		\bd{Student Number:} \ @00090847 \hspace{\fill} \bd{Assignment:} 3 \\
		\hrule	
	
	\section*{Problem 1: Exercise 5.3 \& 5.4}
	
	If $\hat{L}_\pm$ and $\hat{E}_\pm$ are defined by $\hat{L}_\pm = \hat{L}_x \pm i\hat{L}_y$ and $\hat{R}_\pm = \hat{X}_x \pm i\hat{Y}_y$, prove the following commutators:
	\begin{enumerate}
		\item[(a)] $[\hat{L}_\pm, \hat{R}_\pm] = \pm 2\hbar \hat{Z}$,
		\paragraph{Solution} The book seems to be wrong for the first two parts.
		\begin{align}
			[\hat{L}_\pm, \hat{R}_\pm] &= [\hat{L}_x \pm i\hat{L}_y, \hat{X}_x \pm i\hat{Y}_y] \\
									   &= [\hat{L}_x,\hat{X}_x] \pm [\hat{L}_x,i\hat{Y}_y] \pm [i\hat{L}_y,\hat{X}_x] \pm [i\hat{L}_y,i\hat{Y}_y] \\
									   &= [\hat{L}_x,\hat{X}_x] \pm i[\hat{L}_x,\hat{Y}_y] \pm i[\hat{L}_y,\hat{X}_x] \mp [\hat{L}_y,\hat{Y}_y] \\
									   &= 0 \pm i(i\hbar\hat{Z}_z) \pm i(-i\hbar\hat{Z}_z) \mp 0 \\
									   &= \pm i^2\hbar\hat{Z}_z \mp i^2\hbar\hat{Z}_z \\
									   &= \boxed{0}
		\end{align}	
		
		\item[(b)] $[\hat{L}_\pm, \hat{R}_\mp] = 0$,
		\paragraph{Solution}
		\begin{align}
			[\hat{L}_\pm, \hat{R}_\mp] &= [\hat{L}_x \pm i\hat{L}_y, \hat{X}_x \mp i\hat{Y}_y] \\
									   &= [\hat{L}_x,\hat{X}_x] \mp [\hat{L}_x,i\hat{Y}_y] \pm [i\hat{L}_y,\hat{X}_x] \mp [i\hat{L}_y,i\hat{Y}_y] \\
									   &= [\hat{L}_x,\hat{X}_x] \mp i[\hat{L}_x,\hat{Y}_y] \pm i[\hat{L}_y,\hat{X}_x] \pm [\hat{L}_y,\hat{Y}_y] \\
									   &= 0 \mp i(i\hbar\hat{Z}_z) \pm i(-i\hbar\hat{Z}_z) \pm 0 \\
									   &= \mp i^2\hbar\hat{Z}_z \mp i^2\hbar\hat{Z}_z \\
									   &= \boxed{\pm 2\hbar\hat{Z}_z}
		\end{align}
		
		\item[(c)] $[\hat{L}_z, \hat{R}_\pm] = \pm \hbar \hat{R}_\pm$,
		\paragraph{Solution} 
		\begin{align}
			[\hat{L}_z, \hat{R}_\pm] &= [\hat{L}_z, \hat{X}_x \pm i\hat{Y}_y] \\
									 &= [\hat{L}_z,\hat{X}_x] \pm [\hat{L}_z,i\hat{Y}_y] \\
									 &= (i\hbar\hat{Y}_y) \pm i(-i\hbar\hat{X}_x) \\
									 &= \pm \hbar\hat{X}_x + i\hbar\hat{Y}_y \\
									 \intertext{We can factor put the $\pm\hbar$, factoring out the $\pm$ from the second term causes it turn to $\pm$.}
									 &= \pm \hbar ( \hat{X}_x \pm i\hat{Y}_y ) \\
									 &= \boxed{\pm \hbar \hat{R}_\pm}
		\end{align}
		
		
		
		\item[(d)] $[\hat{L}_z, \hat{Z}] = 0$,
		\paragraph{Solution} sol
		\begin{align}
			[\hat{L}_z, \hat{Z}] &= [\hat{X} \hat{P}_y - \hat{Y} \hat{P}_x, \hat{Z}] \\
								 &= [\hat{X} \hat{P}_y, \hat{Z}] - [\hat{Y} \hat{P}_x, \hat{Z}] \\
								 &= \hat{X}[\hat{P}_y, \hat{Z}] - \hat{Y}[\hat{P}_x, \hat{Z}] \\
								 &= \boxed{0}
		\end{align}
		
	\end{enumerate}	
	
\clearpage
	
	\section*{Problem 2: Exercise 5.11}
	
	Consider the wave function
	\begin{equation} \label{eq:1}
		\Psi(\theta, \phi) = 3\sin \theta \cos \theta e^{i\phi} - 2\left( 1 - \cos^2 \theta \right) e^{2i\phi}. 
	\end{equation}
	\begin{enumerate}
		\item[(a)] Write $\Psi(\theta, \phi)$ in terms of the spherical harmonics.
		\paragraph{Solution} We can find the spherical harmonics in the book, by looking at our wave function, I single out the relevant equations. 
		\begin{align}
			Y_{2,\pm1} (\theta, \phi) &= \mp \sqrt{\frac{15}{8\pi}} e^{\pm i\phi} \sin \theta \cos \theta \\
			Y_{2,\pm2} (\theta, \phi) &= \sqrt{\frac{15}{32\pi}} e^{\pm 2i\phi} \sin^2 \theta
		\end{align}
		\begin{equation}
			\begin{aligned}[b]
				\Psi(\theta, \phi) &= 3\sin \theta \cos \theta e^{i\phi} - 2\sin^2 \theta e^{2i\phi} \\
								   &= \boxed{-3 \sqrt{\frac{8\pi}{15}} Y_{2,1} (\theta, \phi) - 2 \sqrt{\frac{32\pi}{15}} Y_{2,2} (\theta, \phi)}
			\end{aligned}
		\end{equation}
		
		\item[(b)] Is $\Psi(\theta, \phi)$ an eigenstate of $\hat{L}^2$ or $\hat{L}_z$?
		\paragraph{Solution} A wave function is an eigenstate of an operator if it satisfies $\hat{A} \Psi = \lambda \Psi$, where $\lambda$ is a constant. $\hat{L}^2\ket{l,m} = \hbar^2 l(l+1)\ket{l,m}$, $\hat{L}_z\ket{l,m} = m\hbar\ket{l,m}$ 
		\begin{equation}
			\begin{aligned}[b]
				\hat{L}^2 \Psi(\theta, \phi) &= -3 \sqrt{\frac{8\pi}{15}} \left[ 2(2+1)\hbar^2 \right] Y_{2,1} - 2 \sqrt{\frac{32\pi}{15}} \left[ 2(2+1)\hbar^2 \right] Y_{2,2} \\
											 &= 6\hbar^2 \Psi(\theta,\phi) \quad \boxed{\text{Eigenstate}}
			\end{aligned}
		\end{equation}
		\begin{equation}
			\begin{aligned}[b]
				\hat{L}_z \Psi(\theta, \phi) &= -3 \sqrt{\frac{8\pi}{15}} \left[ 1 \times \hbar \right] Y_{2,1} - 2 \sqrt{\frac{32\pi}{15}} \left[ 2 \times \hbar \right] Y_{2,2} \\
										   	 &\neq \lambda \Psi(\theta,\phi) \quad \boxed{\text{Not an eigenstate}}
			\end{aligned}
		\end{equation}
		
		\item[(b)] Find the probability of measuring $2\hbar$ for the $z$-component of the orbital angular momentum. 
		\paragraph{Solution} The only state with $2\hbar$ is $Y_{2,2}$.
		\begin{equation}
			\begin{aligned}[b]
				P_2 &= \abs{ \braket{2,2}{\Psi} }^2 \\
					&= \abs{ -3 \sqrt{\frac{8\pi}{15}} \braket{2,2}{2,1} - 2 \sqrt{\frac{32\pi}{15}} \braket{2,2}{2,2} }^2 \\
					&= \abs{-2 \sqrt{\frac{32\pi}{15}}}^2 \\
					&= \boxed{\frac{128\pi}{15}}
			\end{aligned}
		\end{equation}
	\end{enumerate}
	
\clearpage
	
	\section*{Problem 3: Exercise 5.16}
	
	Consider a system which is described by the state
	\begin{equation} \label{eq:2}
		\Psi(\theta, \phi) = \sqrt{\frac{3}{8}} Y_{1,1}(\theta, \phi) + \sqrt{\frac{1}{8}} Y_{1,0}(\theta, \phi) + A Y_{1,-1}(\theta, \phi), 
	\end{equation}
	where $A$ is a real constant.
	\begin{enumerate}
		\item[(a)] Calculate $A$ so that $\ket{\Psi}$ is normalized.
		\paragraph{Solution}
		\begin{equation}
			\begin{aligned}[b]
				\braket{\Psi}{\Psi} = \frac{3}{8} + \frac{1}{8} + A^2 &= 1 \\
																	A &= \sqrt{\frac{1}{2}}
			\end{aligned}
		\end{equation}
		
		\item[(b)] Find $\hat{L}_+ \Psi(\theta, \phi)$.
		\paragraph{Solution} $\hat{L}_{\pm}\ket{l,m} = \hbar \sqrt{l(l+1)-m(m \pm 1)} \ket{l,m \pm 1}$.
		\begin{equation}
			\begin{aligned}[b]
				\hat{L}_+ \Psi(\theta, \phi) &= \sqrt{\frac{3}{8}} \hat{L}_+ \ket{1,1} + \sqrt{\frac{1}{8}} \hat{L}_+ \ket{1,0} + \sqrt{\frac{1}{2}} \hat{L}_+ \ket{1,-1} \\
											 &= \sqrt{\frac{3}{8}} \hbar \sqrt{ 1(1+1) - 1(1+1) } \ket{1,2} \\
											 &\quad + \sqrt{\frac{1}{8}} \hbar \sqrt{ 1(1+1) - 0(0+1) } \ket{1,1} \\
											 &\quad + \sqrt{\frac{1}{2}} \hbar \sqrt{ 1(1+1) + 1(-1+1) } \ket{1,0} \\
											 &= 0 + \frac{\hbar}{2} \ket{1,1} + \hbar \ket{1,0} \\
											 &= \boxed{\hbar \left(\frac{1}{2} \ket{1,1} + \ket{1,0}\right)}
			\end{aligned}
		\end{equation}
		
		\item[(c)] Calculate the expectation values of $\hat{L}_x$ and $\hat{L}^2$ in the state $\ket{\Psi}$.
		\paragraph{Solution}
		
		\begin{equation}
			\begin{split}
				\hat{L}_\pm = \hat{L}_x \pm i\hat{l}_y \quad \Rightarrow \quad \hat{L}_x = \frac{1}{2} \left[ \hat{L}_+ + \hat{L}_- \right]
			\end{split}
		\end{equation}
		
		\begin{equation}
			\begin{aligned}[b]
				\expval{\hat{L}_+} &= \left[ \sqrt{\frac{3}{8}} \ket{1,1} + \sqrt{\frac{1}{8}} \ket{1,0} + \sqrt{\frac{1}{2}} \ket{1,-1} \right] \left[ \sqrt{\frac{3}{8}} (0) \ket{1,2} + \sqrt{\frac{1}{8}} \sqrt{2}\hbar \ket{1,1} + \sqrt{\frac{1}{2}} \sqrt{2}\hbar \ket{1,0} \right] \\
								   &= \hbar \left[ \sqrt{\frac{3}{8} \times \frac{1}{8} \times 2} + \sqrt{\frac{1}{8} \times \frac{1}{2} \times 2} \right] \\
								   &= \frac{\sqrt{6} + 2\sqrt{2}}{8} \hbar
			\end{aligned}
		\end{equation}
		
		\begin{equation}
			\begin{aligned}[b]
				\expval{\hat{L}_-} &= \left[ \sqrt{\frac{3}{8}} \ket{1,1} + \sqrt{\frac{1}{8}} \ket{1,0} + \sqrt{\frac{1}{2}} \ket{1,-1} \right] \left[ \sqrt{\frac{3}{8}} \sqrt{2}\hbar \ket{1,0} + \sqrt{\frac{1}{8}} \sqrt{2}\hbar \ket{1,-1} + \sqrt{\frac{1}{2}} (0) \ket{1,-2} \right] \\
								   &= \hbar \left[ \sqrt{\frac{1}{8} \times \frac{3}{8} \times 2} + \sqrt{\frac{1}{2} \times \frac{1}{8} \times 2} \right] \\
								   &= \frac{\sqrt{6} + 2\sqrt{2}}{8} \hbar
			\end{aligned}
		\end{equation}
		
		\begin{equation}
			\begin{aligned}[b]
				\hat{L}_x &= \frac{1}{2} \left[ \frac{\sqrt{6} + 2\sqrt{2}}{8} \hbar + \frac{\sqrt{6} + 2\sqrt{2}}{8} \hbar \right] \\
						  &= \boxed{\frac{\sqrt{6} + 2\sqrt{2}}{8} \hbar} \\
			\end{aligned}
		\end{equation}
		
		\begin{equation}
			\begin{aligned}[b]
				\expval{\hat{L}^2} &= \left[ \sqrt{\frac{3}{8}} \ket{1,1} + \sqrt{\frac{1}{8}} \ket{1,0} + \sqrt{\frac{1}{2}} \ket{1,-1} \right] \left[ \sqrt{\frac{3}{8}} \hat{L}^2 \ket{1,1} + \sqrt{\frac{1}{8}} \hat{L}^2 \ket{1,0} + \sqrt{\frac{1}{2}} \hat{L}^2 \ket{1,-1} \right] \\
								   &= \left[ \sqrt{\frac{3}{8}} \ket{1,1} + \sqrt{\frac{1}{8}} \ket{1,0} + \sqrt{\frac{1}{2}} \ket{1,-1} \right] \left[ \sqrt{\frac{3}{8}} 2\hbar^2 \ket{1,1} + \sqrt{\frac{1}{8}} 2\hbar^2 \ket{1,0} + \sqrt{\frac{1}{2}} 2\hbar^2 \ket{1,-1} \right] \\
								   &= 2\hbar^2 \left[ \frac{3}{8} + \frac{1}{8} + \frac{1}{2} \right] \\
								   &= \boxed{2\hbar^2} 
			\end{aligned}
		\end{equation}
		
		\item[(d)] Find the probability associated with a measurement that gives zero for the $z$-component of the angular momentum.
		\paragraph{Solution}
		
		\begin{equation}
			\begin{aligned}[b]
				\hat{L}_z \Psi = \sqrt{\frac{3}{8}} (1 \times \hbar) \ket{1,1} + \sqrt{\frac{1}{8}} (0 \times \hbar) \ket{1,0} + \sqrt{\frac{1}{2}} (-1 \times \hbar) \ket{1,-1}
			\end{aligned}
		\end{equation}
		The only state with $0$ is $Y_{1,0}$.
		\begin{equation}
			\begin{aligned}[b]
				P_0 &= \abs{ \braket{1,0}{\Psi} }^2 \\
					&= \abs{ \sqrt{\frac{3}{8}} \bra{1,0}\ket{1,1} + \sqrt{\frac{1}{8}} \bra{1,0}\ket{1,0} + \sqrt{\frac{1}{2}} \bra{1,0}\ket{1,-1} }^2 \\
					&= \abs{ \sqrt{\frac{1}{8}} }^2 \\
					&= \boxed{\frac{1}{8}}
			\end{aligned}
		\end{equation}
		
		\item[(e)] Calculate $\bra{\Phi}\hat{L}_z \ket{\Psi}$ and $\bra{\Phi}\hat{L}_- \ket{\Psi}$ where
		\begin{equation} \label{eq:3}
			\Phi(\theta, \phi) = \sqrt{\frac{8}{15}} Y_{1,1}(\theta, \phi) + \sqrt{\frac{4}{15}} Y_{1,0}(\theta, \phi) + \sqrt{\frac{3}{15}} Y_{2,-1}(\theta, \phi). 
		\end{equation}
		\paragraph{Solution}
		\begin{equation}
			\begin{aligned}[b]
				\bra{\Phi}\hat{L}_z \ket{\Psi} &= \left[ \sqrt{\frac{8}{15}} \ket{1,1} + \sqrt{\frac{4}{15}} \ket{1,0} + \sqrt{\frac{3}{15}} \ket{2,-1} \right] \left[ \sqrt{\frac{3}{8}} \hbar \ket{1,1} - \sqrt{\frac{1}{2}} \hbar \ket{1,-1} \right] \\
											   &= \hbar \left[ \sqrt{\frac{8}{15}} \times \sqrt{\frac{3}{8}} \right] \\
											   &= \boxed{\frac{\sqrt{5}}{5}\hbar}
			\end{aligned}
		\end{equation}
		\begin{equation}
			\begin{aligned}[b]
				\bra{\Phi}\hat{L}_- \ket{\Psi} &= \left[ \sqrt{\frac{8}{15}} \ket{1,1} + \sqrt{\frac{4}{15}} \ket{1,0} + \sqrt{\frac{3}{15}} \ket{2,-1} \right] \left[ \sqrt{\frac{3}{8}} \sqrt{2}\hbar \ket{1,0} + \sqrt{\frac{1}{8}} \sqrt{2}\hbar \ket{1,-1} \right] \\
											   &= \hbar \left[ \sqrt{\frac{4}{15}} \times \sqrt{\frac{3}{8}} \times \sqrt{2} \right] \\
											   &= \boxed{\frac{\sqrt{5}}{5}\hbar}
			\end{aligned}
		\end{equation}
	\end{enumerate}
	
\clearpage

	\section*{Problem 4: Exercise 5.28}
	
	Consider a system which is given in the following angular momentum eigenstates $\ket{l,m}$:
	\begin{equation} \label{eq:4}
		\ket{\Psi} = \sqrt{\frac{1}{7}} \ket{1,-1} + A\ket{1,0} + \sqrt{\frac{2}{7}} \ket{1,1}, 
	\end{equation}
	where $A$ is a real constant.
	\begin{enumerate}
		\item[(a)] Calculate $A$ so that $\ket{\Psi}$ is normalized.
		\paragraph{Solution}
		\begin{equation}
			\begin{aligned}[b]
				\braket{\Psi}{\Psi} = \frac{1}{7} + A^2 + \frac{2}{7} &= 1 \\
																	A &= \sqrt{\frac{4}{7}}
			\end{aligned}
		\end{equation}
		
		\item[(b)] Calculate the expectation values of $\hat{L}_x$, $\hat{L}_y$, $\hat{L}_z$, and $\hat{L}^2$ in the state $\ket{\Psi}$.
		\paragraph{Solution}
		
		\begin{equation}
			\begin{aligned}[b]
				\expval{\hat{L}^2} &= \left[ \sqrt{\frac{1}{7}} \ket{1,-1} + \sqrt{\frac{4}{7}} \ket{1,0} + \sqrt{\frac{5}{7}} \ket{1,1} \right] \left[ 2\hbar^2 \sqrt{\frac{1}{7}} \ket{1,-1} + 2\hbar^2 \sqrt{\frac{4}{7}} \ket{1,0} + 2\hbar^2 \sqrt{\frac{2}{7}} \ket{1,1} \right] \\
								   &= 2\hbar^2 \left[ \frac{1}{7} + \frac{4}{7} + \frac{2}{7} \right] \\
								   &= \boxed{2\hbar^2} 
			\end{aligned}
		\end{equation}
		\begin{equation}
			\begin{aligned}[b]
				\expval{\hat{L}_z} &= \left[ \sqrt{\frac{1}{7}} \ket{1,-1} + \sqrt{\frac{4}{7}} \ket{1,0} + \sqrt{\frac{5}{7}} \ket{1,1} \right] \left[ \sqrt{\frac{1}{7}} (-1 \times \hbar) \ket{1,-1} + \sqrt{\frac{4}{7}} (0 \times \hbar) \ket{1,0} + \sqrt{\frac{2}{7}} (1 \times \hbar) \ket{1,1} \right] \\
								   &= \hbar \left[ \frac{1}{7} + \frac{2}{7} \right] \\
								   &= \boxed{\frac{3}{7} \hbar}
			\end{aligned}
		\end{equation}
		\begin{equation}
			\begin{aligned}[b]
				\expval{\hat{L}_+} &= \left[ \sqrt{\frac{1}{7}} \ket{1,-1} + \sqrt{\frac{4}{7}} \ket{1,0} + \sqrt{\frac{5}{7}} \ket{1,1} \right] \left[ \sqrt{\frac{1}{7}} \sqrt{2}\hbar \ket{1,0} + \sqrt{\frac{4}{7}} \sqrt{2}\hbar \ket{1,1} + \sqrt{\frac{5}{7}} (0) \ket{1,2} \right] \\
								   &= \hbar \left[ \sqrt{\frac{4}{7} \times \frac{1}{7} \times 2} + \sqrt{\frac{5}{7} \times \frac{4}{7} \times 2} \right] \\
								   &= \boxed{\frac{2\sqrt{10} + 2\sqrt{2}}{7} \hbar}
			\end{aligned}
		\end{equation}
		\begin{equation}
			\begin{aligned}[b]
				\expval{\hat{L}_-} &= \left[ \sqrt{\frac{1}{7}} \ket{1,-1} + \sqrt{\frac{4}{7}} \ket{1,0} + \sqrt{\frac{5}{7}} \ket{1,1} \right] \left[ \sqrt{\frac{1}{7}} (0) \ket{1,-2} + \sqrt{\frac{4}{7}} \sqrt{2}\hbar \ket{1,-1} + \sqrt{\frac{5}{7}} \sqrt{2}\hbar \ket{1,0} \right] \\
								   &= \hbar \left[ \sqrt{\frac{1}{7} \times \frac{4}{7} \times 2} + \sqrt{\frac{4}{7} \times \frac{5}{7} \times 2} \right] \\
								   &= \boxed{\frac{2\sqrt{10} + 2\sqrt{2}}{7} \hbar}
			\end{aligned}
		\end{equation}
		
		\begin{minipage}{0.45\textwidth}
			\begin{equation}
				\begin{split}
					\expval{\hat{L}_x} &= \frac{1}{2} \left[ \frac{2\sqrt{10} + 2\sqrt{2}}{7} \hbar + \frac{2\sqrt{10} + 2\sqrt{2}}{7} \hbar \right] \\
									   &= \boxed{\frac{2\sqrt{10} + 2\sqrt{2}}{7} \hbar}
				\end{split}
			\end{equation}
		\end{minipage}
		\begin{minipage}{0.45\textwidth}
			\begin{equation}
				\begin{split}
					\expval{\hat{L}_y} &= \frac{1}{2} \left[ \frac{2\sqrt{10} + 2\sqrt{2}}{7} \hbar - \frac{2\sqrt{10} + 2\sqrt{2}}{7} \hbar \right] \\
									   &= \boxed{0}
				\end{split}
			\end{equation}
		\end{minipage}
		
		\item[(c)] Find the probability associated with a measurement that gives $1\hbar$ for the $z$-component of the angular momentum.
		\paragraph{Solution} The only state with $1\hbar$ is $Y_{1,1}$.
		\begin{equation}
			\begin{aligned}[b]
				P_1 &= \abs{ \braket{1,1}{\Psi} }^2 \\
					&= \abs{ \sqrt{\frac{1}{7}} \bra{1,1}\ket{1,-1} + \sqrt{\frac{4}{7}} \bra{1,1}\ket{1,0} + \sqrt{\frac{5}{7}} \bra{1,1}\ket{1,1} }^2 \\
					&= \abs{ \sqrt{\frac{5}{7}} }^2 \\
					&= \boxed{\frac{5}{7}}
			\end{aligned}
		\end{equation}
		
		\item[(d)] Calculate $\bra{1,m}\hat{L}^2_+ \ket{\Psi}$ and $\bra{1,m}\hat{L}^2_- \ket{\Psi}$.
		\paragraph{Solution} We need to apply the operators twice, then multiply by the bra.
		\begin{equation}
			\begin{split}
				\hat{L}_+ \ket{\Psi} &= \sqrt{\frac{1}{7}} \sqrt{2}\hbar \ket{1,0} + \sqrt{\frac{4}{7}} \sqrt{2}\hbar \ket{1,1}
			\end{split}
		\end{equation}
		\begin{equation}
			\begin{aligned}[b]
				\hat{L}^2_+ \ket{\Psi} &= \sqrt{\frac{1}{7}} \sqrt{2}\hbar \sqrt{2}\hbar \ket{1,1} + \sqrt{\frac{4}{7}} \sqrt{2}\hbar (0) \ket{1,2} \\
				&= \sqrt{\frac{1}{7}} 2\hbar^2 \ket{1,1}
			\end{aligned}
		\end{equation}
		\begin{equation}
			\begin{aligned}[b]
				\bra{1,m}\hat{L}^2_+ \ket{\Psi} &= \sqrt{\frac{4}{7}} \hbar^2 \bra{1,m}\ket{1,1} \\
												&= \boxed{\sqrt{\frac{4}{7}} \hbar^2 \delta_{m,1}}
			\end{aligned}
		\end{equation}
		
		\begin{equation}
			\begin{split}
				\hat{L}_- \ket{\Psi} &= \sqrt{\frac{4}{7}} \sqrt{2}\hbar \ket{1,-1} + \sqrt{\frac{5}{7}} \sqrt{2}\hbar \ket{1,0}
			\end{split}
		\end{equation}
		\begin{equation}
			\begin{aligned}[b]
				\hat{L}^2_- \ket{\Psi} &= \sqrt{\frac{4}{7}} \sqrt{2}\hbar (0) \ket{1,-2} + \sqrt{\frac{5}{7}} \sqrt{2}\hbar \sqrt{2}\hbar \ket{1,-1} \\
				&= \sqrt{\frac{5}{7}} 2\hbar^2 \ket{1,-1}
			\end{aligned}
		\end{equation}
		\begin{equation}
			\begin{aligned}[b]
				\bra{1,m}\hat{L}^2_- \ket{\Psi} &= \sqrt{\frac{20}{7}} \hbar^2 \bra{1,m}\ket{1,-1} \\
				&= \sqrt{\frac{20}{7}} \hbar^2 \delta_{m,-1}
			\end{aligned}
		\end{equation}
	\end{enumerate}
	
	
\clearpage

	\section*{Problem 5: Exercise 5.31}
	
	Consider a spin $\frac{3}{2}$ particle whose Hamiltonian is given by
	\begin{equation} \label{eq:5}
		\hat{H} = \frac{\varepsilon_0}{\hbar^2} ( \hat{S}^2_x - \hat{S}^2_y ) - \frac{\varepsilon_0}{\hbar^2} ( \hat{S}^2_z ) 
	\end{equation}
	where $\epsilon_0$ is a constant having the dimensions of energy.
	\begin{enumerate}
		\item[(a)] Find the matrix of the Hamiltonian and diagonalize it to find the energy levels.
		\paragraph{Solution} Need to find the matrix for $\hat{S}_x, \hat{S}_y,$ and $\hat{S}_z$. For long matrix calculations I used \href{https://matrixcalc.org/}{https://matrixcalc.org/}.
		
		\begingroup
		\renewcommand*{\arraystretch}{1.5}
		\begin{equation}
			\hat{S}_z =
			\begin{bmatrix}
				\bra{\frac{3}{2},\frac{3}{2}} \frac{3}{2}\hbar \ket{\frac{3}{2},\frac{3}{2}} & \bra{\frac{3}{2},\frac{3}{2}} \frac{1}{2}\hbar \ket{\frac{3}{2},\frac{1}{2}} &
				\bra{\frac{3}{2},\frac{3}{2}} \frac{-1}{2}\hbar \ket{\frac{3}{2},\frac{-1}{2}} & \bra{\frac{3}{2},\frac{3}{2}} \frac{-3}{2}\hbar \ket{\frac{3}{2},\frac{-3}{2}} \\
				\bra{\frac{3}{2},\frac{1}{2}} \frac{3}{2}\hbar \ket{\frac{3}{2},\frac{3}{2}} & \bra{\frac{3}{2},\frac{1}{2}} \frac{1}{2}\hbar \ket{\frac{3}{2},\frac{1}{2}} &
				\bra{\frac{3}{2},\frac{1}{2}} \frac{-1}{2}\hbar \ket{\frac{3}{2},\frac{-1}{2}} & \bra{\frac{3}{2},\frac{1}{2}} \frac{-3}{2}\hbar \ket{\frac{3}{2},\frac{-3}{2}} \\
				\bra{\frac{3}{2},\frac{-1}{2}} \frac{3}{2}\hbar \ket{\frac{3}{2},\frac{3}{2}} & \bra{\frac{3}{2},\frac{-1}{2}} \frac{1}{2}\hbar \ket{\frac{3}{2},\frac{1}{2}} &
				\bra{\frac{3}{2},\frac{-1}{2}} \frac{-1}{2}\hbar \ket{\frac{3}{2},\frac{-1}{2}} & \bra{\frac{3}{2},\frac{-1}{2}} \frac{-3}{2}\hbar \ket{\frac{3}{2},\frac{-3}{2}} \\
				\bra{\frac{3}{2},\frac{-3}{2}} \frac{3}{2}\hbar \ket{\frac{3}{2},\frac{3}{2}} & \bra{\frac{3}{2},\frac{-3}{2}} \frac{1}{2}\hbar \ket{\frac{3}{2},\frac{1}{2}} &
				\bra{\frac{3}{2},\frac{-3}{2}} \frac{-1}{2}\hbar \ket{\frac{3}{2},\frac{-1}{2}} & \bra{\frac{3}{2},\frac{-3}{2}} \frac{-3}{2}\hbar \ket{\frac{3}{2},\frac{-3}{2}} \\
			\end{bmatrix}
		\end{equation}
		Applying the orthogonality condition, only the diagonal terms will survive.
		\begin{equation}
			\hat{S}_z = \frac{\hbar}{2}
			\begin{bmatrix}
				3 & 0 & 0 & 0 \\
				0 & 1 & 0 & 0 \\
				0 & 0 &	-1 & 0 \\
				0 & 0 & 0 & -3 \\
			\end{bmatrix}
		\end{equation}
		\begin{equation}
			\hat{S}_z = \frac{\hbar^2}{4}
			\begin{bmatrix}
				9 & 0 & 0 & 0 \\
				0 & 1 & 0 & 0 \\
				0 & 0 &	1 & 0 \\
				0 & 0 & 0 & 9 \\
			\end{bmatrix}
		\end{equation}
		For $\hat{S}_x$ and $\hat{S}_y$, the ladder operators $\hat{S}_\pm$ need to be found first. I will skip adding the $s$ quantum number in the matrix for simplicity. Also the $a_{mn}$ denotes the constant from $\hat{S}_\pm$.
		\begin{align}
			\hat{S}_+ &=
			\begin{bmatrix}
				\bra{\frac{3}{2}} \hat{S}_+ \ket{\frac{3}{2}} & \bra{\frac{3}{2}} \hat{S}_+ \ket{\frac{1}{2}} &
				\bra{\frac{3}{2}} \hat{S}_+ \ket{\frac{-1}{2}} & \bra{\frac{3}{2}} \hat{S}_+ \ket{\frac{-3}{2}} \\
				\bra{\frac{1}{2}} \hat{S}_+ \ket{\frac{3}{2}} & \bra{\frac{1}{2}} \hat{S}_+ \ket{\frac{1}{2}} &
				\bra{\frac{1}{2}} \hat{S}_+ \ket{\frac{-1}{2}} & \bra{\frac{1}{2}} \hat{S}_+ \ket{\frac{-3}{2}} \\
				\bra{\frac{-1}{2}} \hat{S}_+ \ket{\frac{3}{2}} & \bra{\frac{-1}{2}} \hat{S}_+ \ket{\frac{1}{2}} &
				\bra{\frac{-1}{2}} \hat{S}_+ \ket{\frac{-1}{2}} & \bra{\frac{-1}{2}} \hat{S}_+ \ket{\frac{-3}{2}} \\
				\bra{\frac{-3}{2}} \hat{S}_+ \ket{\frac{3}{2}} & \bra{\frac{-3}{2}} \hat{S}_+ \ket{\frac{1}{2}} &
				\bra{\frac{-3}{2}} \hat{S}_+ \ket{\frac{-1}{2}} & \bra{\frac{-3}{2}} \hat{S}_+ \ket{\frac{-3}{2}} \\
			\end{bmatrix} \\
			&=
			\begin{bmatrix}
				\bra{\frac{3}{2}} a_{11} \ket{\frac{5}{2}} & \bra{\frac{3}{2}} a_{12} \ket{\frac{3}{2}} &
				\bra{\frac{3}{2}} a_{13} \ket{\frac{1}{2}} & \bra{\frac{3}{2}} a_{14} \ket{\frac{-1}{2}} \\
				\bra{\frac{1}{2}} a_{21} \ket{\frac{5}{2}} & \bra{\frac{1}{2}} a_{22} \ket{\frac{3}{2}} &
				\bra{\frac{1}{2}} a_{23} \ket{\frac{1}{2}} & \bra{\frac{1}{2}} a_{24} \ket{\frac{-1}{2}} \\
				\bra{\frac{-1}{2}} a_{31} \ket{\frac{5}{2}} & \bra{\frac{-1}{2}} a_{32} \ket{\frac{3}{2}} &
				\bra{\frac{-1}{2}} a_{33} \ket{\frac{1}{2}} & \bra{\frac{-1}{2}} a_{34} \ket{\frac{-1}{2}} \\
				\bra{\frac{-3}{2}} a_{41} \ket{\frac{5}{2}} & \bra{\frac{-3}{2}} a_{42} \ket{\frac{3}{2}} &
				\bra{\frac{-3}{2}} a_{43} \ket{\frac{1}{2}} & \bra{\frac{-3}{2}} a_{44} \ket{\frac{-1}{2}} \\
			\end{bmatrix} \\
			&=
			\begin{bmatrix}
				0 & \bra{\frac{3}{2}} a_{12} \ket{\frac{3}{2}} & 0 & 0 \\
				0 & 0 & \bra{\frac{1}{2}} a_{23} \ket{\frac{1}{2}} & 0 \\
				0 & 0 & 0 & \bra{\frac{-1}{2}} a_{34} \ket{\frac{-1}{2}} \\
				0 & 0 & 0 & 0 \\
			\end{bmatrix} \\
			\hat{S}_+ &= \frac{\hbar}{2}
			\begin{bmatrix}
				0 & \sqrt{3} & 0 & 0 \\
				0 & 0 & 2 & 0 \\
				0 & 0 & 0 & \sqrt{3} \\
				0 & 0 & 0 & 0 \\
			\end{bmatrix}
		\end{align}
		
		\begin{align}
			\hat{S}_- &=
			\begin{bmatrix}
				\bra{\frac{3}{2}} \hat{S}_- \ket{\frac{3}{2}} & \bra{\frac{3}{2}} \hat{S}_- \ket{\frac{1}{2}} &
				\bra{\frac{3}{2}} \hat{S}_- \ket{\frac{-1}{2}} & \bra{\frac{3}{2}} \hat{S}_- \ket{\frac{-3}{2}} \\
				\bra{\frac{1}{2}} \hat{S}_- \ket{\frac{3}{2}} & \bra{\frac{1}{2}} \hat{S}_- \ket{\frac{1}{2}} &
				\bra{\frac{1}{2}} \hat{S}_- \ket{\frac{-1}{2}} & \bra{\frac{1}{2}} \hat{S}_- \ket{\frac{-3}{2}} \\
				\bra{\frac{-1}{2}} \hat{S}_- \ket{\frac{3}{2}} & \bra{\frac{-1}{2}} \hat{S}_- \ket{\frac{1}{2}} &
				\bra{\frac{-1}{2}} \hat{S}_- \ket{\frac{-1}{2}} & \bra{\frac{-1}{2}} \hat{S}_- \ket{\frac{-3}{2}} \\
				\bra{\frac{-3}{2}} \hat{S}_- \ket{\frac{3}{2}} & \bra{\frac{-3}{2}} \hat{S}_- \ket{\frac{1}{2}} &
				\bra{\frac{-3}{2}} \hat{S}_- \ket{\frac{-1}{2}} & \bra{\frac{-3}{2}} \hat{S}_- \ket{\frac{-3}{2}} \\
			\end{bmatrix} \\
			&=
			\begin{bmatrix}
				\bra{\frac{3}{2}} a_{11} \ket{\frac{1}{2}} & \bra{\frac{3}{2}} a_{12} \ket{\frac{-1}{2}} &
				\bra{\frac{3}{2}} a_{13} \ket{\frac{-3}{2}} & \bra{\frac{3}{2}} a_{14} \ket{\frac{-5}{2}} \\
				\bra{\frac{1}{2}} a_{21} \ket{\frac{1}{2}} & \bra{\frac{1}{2}} a_{22} \ket{\frac{-1}{2}} &
				\bra{\frac{1}{2}} a_{23} \ket{\frac{-3}{2}} & \bra{\frac{1}{2}} a_{24} \ket{\frac{-5}{2}} \\
				\bra{\frac{-1}{2}} a_{31} \ket{\frac{1}{2}} & \bra{\frac{-1}{2}} a_{32} \ket{\frac{-1}{2}} &
				\bra{\frac{-1}{2}} a_{33} \ket{\frac{-3}{2}} & \bra{\frac{-1}{2}} a_{34} \ket{\frac{-5}{2}} \\
				\bra{\frac{-3}{2}} a_{41} \ket{\frac{1}{2}} & \bra{\frac{-3}{2}} a_{42} \ket{\frac{-1}{2}} &
				\bra{\frac{-3}{2}} a_{43} \ket{\frac{-3}{2}} & \bra{\frac{-3}{2}} a_{44} \ket{\frac{-5}{2}} \\
			\end{bmatrix} \\
			&=
			\begin{bmatrix}
				0 & 0 & 0 & 0 \\
				\bra{\frac{1}{2}} a_{21} \ket{\frac{1}{2}} & 0 & 0 & 0 \\
				0 & \bra{\frac{-1}{2}} a_{32} \ket{\frac{-1}{2}} & 0 & 0 \\
				0 & 0 & \bra{\frac{-3}{2}} a_{43} \ket{\frac{-3}{2}} & 0 \\
			\end{bmatrix} \\
			\hat{S}_- &= \frac{\hbar}{2}
			\begin{bmatrix}
				0 & 0 & 0 & 0 \\
				\sqrt{3} & 0 & 0 & 0 \\
				0 & 2 & 0 & 0 \\
				0 & 0 & \sqrt{3} & 0 \\
			\end{bmatrix}
		\end{align}
		\begin{minipage}{0.45\textwidth}
			\begin{equation}
				\hat{S}_x = \frac{\hbar}{2}
				\begin{bmatrix}
					0 & \sqrt{3} & 0 & 0 \\
					\sqrt{3} & 0 & 2 & 0 \\
					0 & 2 & 0 & \sqrt{3} \\
					0 & 0 & \sqrt{3} & 0 \\
				\end{bmatrix}
			\end{equation}
		\end{minipage}
		\begin{minipage}{0.45\textwidth}
			\begin{equation}
				\hat{S}_y = \frac{i\hbar}{2}
				\begin{bmatrix}
					0 & -\sqrt{3} & 0 & 0 \\
					\sqrt{3} & 0 & -2 & 0 \\
					0 & 2 & 0 & -\sqrt{3} \\
					0 & 0 & \sqrt{3} & 0 \\
				\end{bmatrix}
			\end{equation}
		\end{minipage}
		
		\begin{minipage}{0.45\textwidth}
			\begin{equation}
				\hat{S}^2_x = \frac{\hbar^2}{4}
				\begin{bmatrix}
					3 & 0 & 2\sqrt{3} & 0 \\
					0 & 7 & 0 & 2\sqrt{3} \\
					2\sqrt{3} & 0 & 7 & 0 \\
					0 & 2\sqrt{3} & 0 & 3 \\
				\end{bmatrix}
			\end{equation}
		\end{minipage}
		\begin{minipage}{0.45\textwidth}
			\begin{equation}
				\hat{S}^2_y = \frac{-\hbar^2}{4}
				\begin{bmatrix}
					-3 & 0 & 2\sqrt{3} & 0 \\
					0 & -7 & 0 & 2\sqrt{3} \\
					2\sqrt{3} & 0 & -7 & 0 \\
					0 & 2\sqrt{3} & 0 & -3 \\
				\end{bmatrix}
			\end{equation}
		\end{minipage}
		\begin{align}
			\hat{H} &= \frac{\varepsilon_0}{4}
			\begin{bmatrix}
				0 & 0 & 4\sqrt{3} & 0 \\
				0 & 0 & 0 & 4\sqrt{3} \\
				4\sqrt{3} & 0 & 0 & 0 \\
				0 & 4\sqrt{3} & 0 & 0
			\end{bmatrix} - \frac{\varepsilon_0}{4}
			\begin{bmatrix}
				9 & 0 & 0 & 0 \\
				0 & 1 & 0 & 0 \\
				0 & 0 &	1 & 0 \\
				0 & 0 & 0 & 9
			\end{bmatrix} \\
			\hat{H} &= \varepsilon_0
			\begin{bmatrix}
				-9 & 0 & 4\sqrt{3} & 0 \\
				0 & -1 & 0 & 4\sqrt{3} \\
				4\sqrt{3} & 0 & -1 & 0 \\
				0 & 4\sqrt{3} & 0 & -9
			\end{bmatrix} 	
		\end{align}
		\endgroup
\clearpage
	
		Now to diagonalize the Hamiltonian, the eigenvalues and the eigenvectors need to be found according to $\hat{H} = PDP^{-1}$, where $P$ is the matrix fo the eigenvectors, and $D$ is a diagonal matrix of the eigenvalues. To find the eigenvalues we find $det(\hat{H} - \lambda \hat{I}) = 0$.
		\begin{align}
			det(\hat{H} - \lambda \hat{I}) &=
			\begin{vmatrix}
				-9 - \lambda & 0 & 4\sqrt{3} & 0 \\
				0 & -1 - \lambda & 0 & 4\sqrt{3} \\
				4\sqrt{3} & 0 & -1 - \lambda & 0 \\
				0 & 4\sqrt{3} & 0 & -9 - \lambda
			\end{vmatrix} = 0 \\
			\intertext{Expanding along the first row.}
			0 &= (-9 - \lambda)
			\begin{vmatrix}
				-1 - \lambda & 0 & 4\sqrt{3} \\
				0 & -1 - \lambda & 0 \\
				4\sqrt{3} & 0 & -9 - \lambda
			\end{vmatrix}
			+ (4\sqrt{3})
			\begin{vmatrix}
				0 & -1 - \lambda & 4\sqrt{3} \\
				4\sqrt{3} & 0 & 0 \\
				0 & 4\sqrt{3} & -9 - \lambda
			\end{vmatrix}
			\intertext{For the first determinant, expand along the second row. The second determinant, along the second row.}
			0 &= (-9 - \lambda) \left[ (-1 - \lambda) \left[ (-1 - \lambda)(-9 - \lambda) - (4\sqrt{3})(4\sqrt{3}) \right] \right] \nonumber \\
			&\quad -(4\sqrt{3})\left[ (4\sqrt{3}) \left[ (-1 - \lambda)(-9 - \lambda) - (4\sqrt{3})(4\sqrt{3}) \right] \right] \\
			0 &= (-9 - \lambda) \left[ (-1 - \lambda) \left[ \lambda^2 + 10\lambda + 9 - 48 \right] \right] \nonumber \\
			&\quad -(4\sqrt{3})\left[ (4\sqrt{3}) \left[ \lambda^2 + 10\lambda + 9 - 48 \right] \right] \\
			0 &= (\lambda^2 + 10\lambda - 39)^2 \\
			0 &= (\lambda - 3)^2(\lambda + 13)^2 \\
			\lambda &= 3, 3, -13, -13
		\end{align}
		With the eigenvalues, we input them back into the original equation $\hat{H} - \lambda \hat{I} = 0$ to find the eigenvectors.
		\begin{align}
			\ket{\lambda_{1,2}} = 3 \quad : \quad
			\begin{pmatrix}
				-12 & 0 & 4\sqrt{3} & 0 \\
				0 & -4 & 0 & 4\sqrt{3} \\
				4\sqrt{3} & 0 & -4 & 0 \\
				0 & 4\sqrt{3} & 0 & -12
			\end{pmatrix}
			\begin{pmatrix}
				a \\
				b \\
				c \\
				d
			\end{pmatrix}
			= \begin{pmatrix}
				0 \\
				0 \\
				0 \\
				0
			\end{pmatrix}
		\end{align}
		\begin{equation}
			\sysdelim..\systeme{
				-12a + 4\sqrt{3}c = 0,
				-4b + 4\sqrt{3}d = 0,
				4\sqrt{3}a - 4c = 0,
				4\sqrt{3}b - 12d = 0
			}
		\end{equation}
		We have $c = \sqrt{3}a$ and $d = \frac{\sqrt{3}}{3}b$, $a$ and $b$ are arbitrary, making a solution set. Set $a=b=1$.
		\begin{equation}
			\ket{\lambda_{1,2}} = A \begin{pmatrix}
				a \\
				b \\
				\sqrt{3}a \\
				\frac{\sqrt{3}}{3}b
			\end{pmatrix}
			= A_1 \begin{pmatrix}
				1 \\
				0 \\
				\sqrt{3} \\
				0
			\end{pmatrix} + A_2 \begin{pmatrix}
				0 \\
				1 \\
				0 \\
				\frac{\sqrt{3}}{3}
			\end{pmatrix}
		\end{equation}
		\begin{equation}
			\begin{split}
				\ket{\lambda_1} &= A_1 \begin{pmatrix}
					1 \\
					0 \\
					\sqrt{3} \\
					0
				\end{pmatrix} \\
				\ket{\lambda_1} &= \frac{1}{2} \begin{pmatrix}
						2 \\
						0 \\
						1
				\end{pmatrix}
			\end{split}
			\qquad \Rightarrow \qquad
			\begin{split}
				\braket{\lambda_1}{\lambda_1} = A_1^2 \begin{pmatrix}
					1 & 0 & \sqrt{3} & 0
				\end{pmatrix}
				\begin{pmatrix}
					1 \\
					0 \\
					\sqrt{3} \\
					0
				\end{pmatrix} &= 1 \\
				A_1^2 \left( 1 + 0 + 3 + 0 \right) &= 1 \\
				A_1 &= \frac{1}{2}
			\end{split}
		\end{equation}
		\noindent\hfil\rule{0.5\textwidth}{.4pt}\hfil \\
		\begin{equation}
			\begin{split}
				\ket{\lambda_2} &= A_2 \begin{pmatrix}
					0 \\
					1 \\
					0 \\
					\frac{\sqrt{3}}{3}
				\end{pmatrix} \\
				\ket{\lambda_2} &= \frac{\sqrt{3}}{2} \begin{pmatrix}
					0 \\
					1 \\
					0 \\
					\frac{\sqrt{3}}{3}
				\end{pmatrix}
			\end{split}
			\qquad \Rightarrow \qquad
			\begin{split}
				\braket{\lambda_2}{\lambda_2} = A_2^2 \begin{pmatrix}
					0 & 1 & 0 & 0 \frac{\sqrt{3}}{3}
				\end{pmatrix}
				\begin{pmatrix}
					0 \\
					1 \\
					0 \\
					\frac{\sqrt{3}}{3}
				\end{pmatrix} &= 1 \\
				A_2^2 \left( 0 + 1 + 0 + \frac{1}{3} \right) &= 1 \\
				A_2 &= \frac{\sqrt{3}}{2}
			\end{split}
		\end{equation}
		\begin{equation}
			\ket{\lambda_{1,2}} = \frac{1}{2}
			\begin{pmatrix}
				1 \\
				0 \\
				\sqrt{3} \\
				0
			\end{pmatrix} + \frac{\sqrt{3}}{2} \begin{pmatrix}
				0 \\
				1 \\
				0 \\
				\frac{\sqrt{3}}{3}
			\end{pmatrix}
		\end{equation}
		
		\begin{align}
			\ket{\lambda_{3,4}} = -13 \quad : \quad
			\begin{pmatrix}
				4 & 0 & 4\sqrt{3} & 0 \\
				0 & 12 & 0 & 4\sqrt{3} \\
				4\sqrt{3} & 0 & 12 & 0 \\
				0 & 4\sqrt{3} & 0 & -4
			\end{pmatrix}
			\begin{pmatrix}
				a \\
				b \\
				c \\
				d
			\end{pmatrix}
			= \begin{pmatrix}
				0 \\
				0 \\
				0 \\
				0
			\end{pmatrix}
		\end{align}
		\begin{equation}
			\sysdelim..\systeme{
				4a + 4\sqrt{3}c = 0,
				12b + 4\sqrt{3}d = 0,
				4\sqrt{3}a + 12c = 0,
				4\sqrt{3}b + 4d = 0
			}
		\end{equation}
		We have $c = -\frac{\sqrt{3}}{3}a$ and $d = -\sqrt{3}b$, $a$ and $b$ are arbitrary, making a solution set. Set $a=b=1$.
		\begin{equation}
			\ket{\lambda_{3,4}} = A \begin{pmatrix}
				a \\
				b \\
				-\frac{\sqrt{3}}{3}a \\
				-\sqrt{3}b
			\end{pmatrix}
			= A_1 \begin{pmatrix}
				1 \\
				0 \\
				-\frac{\sqrt{3}}{3} \\
				0
			\end{pmatrix} + A_2 \begin{pmatrix}
				0 \\
				1 \\
				0 \\
				-\sqrt{3}
			\end{pmatrix}
		\end{equation}
		\begin{equation}
			\begin{split}
				\ket{\lambda_3} &= A_1 \begin{pmatrix}
					1 \\
					0 \\
					-\frac{\sqrt{3}}{3} \\
					0
				\end{pmatrix} \\
				\ket{\lambda_3} &= \frac{\sqrt{3}}{2} \begin{pmatrix}
					1 \\
					0 \\
					-\frac{\sqrt{3}}{3} \\
					0
				\end{pmatrix}
			\end{split}
			\qquad \Rightarrow \qquad
			\begin{split}
				\braket{\lambda_3}{\lambda_3} = A_1^2 \begin{pmatrix}
					1 & 0 & -\frac{\sqrt{3}}{3} & 0
				\end{pmatrix}
				\begin{pmatrix}
					1 \\
					0 \\
					-\frac{\sqrt{3}}{3} \\
					0
				\end{pmatrix} &= 1 \\
				A_1^2 \left( 1 + 0 + \frac{1}{3} + 0 \right) &= 1 \\
				A_1 &= \frac{\sqrt{3}}{2}
			\end{split}
		\end{equation}
		\noindent\hfil\rule{0.5\textwidth}{.4pt}\hfil \\
		\begin{equation}
			\begin{split}
				\ket{\lambda_4} &= A_2 \begin{pmatrix}
					0 \\
					1 \\
					0 \\
					-\sqrt{3}
				\end{pmatrix} \\
				\ket{\lambda_4} &= \frac{1}{2} \begin{pmatrix}
					0 \\
					1 \\
					0 \\
					-\sqrt{3}
				\end{pmatrix}
			\end{split}
			\qquad \Rightarrow \qquad
			\begin{split}
				\braket{\lambda_4}{\lambda_4} = A_2^2 \begin{pmatrix}
					0 & 1 & 0 & 0 -\sqrt{3}
				\end{pmatrix}
				\begin{pmatrix}
					0 \\
					1 \\
					0 \\
					-\sqrt{3}
				\end{pmatrix} &= 1 \\
				A_2^2 \left( 0 + 1 + 0 + 3 \right) &= 1 \\
				A_2 &= \frac{1}{2}
			\end{split}
		\end{equation}
		
		\begin{equation}
			\ket{\lambda_{3,4}} = \frac{\sqrt{3}}{2} \begin{pmatrix}
				1 \\
				0 \\
				-\frac{\sqrt{3}}{3} \\
				0
			\end{pmatrix} + \frac{1}{2} \begin{pmatrix}
				0 \\
				1 \\
				0 \\
				-\sqrt{3}
			\end{pmatrix}
		\end{equation}
		
		\item[(b)] Find the eigenvectors and verify that the energy levels are doubly degenerate.
		\paragraph{Solution} From part (a) the eigenvectors were found, and the fact that each eigenvalue produces two eigenvectors points to the energy levels being doubly degenerate. Explicitly shown in $\hat{H} \Psi = E \Psi$, where $E$ corresponds to the $\lambda$'s that I found.
	\end{enumerate}
	
	
	

\end{document}