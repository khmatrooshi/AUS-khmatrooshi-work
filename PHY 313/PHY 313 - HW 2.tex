\documentclass{article}
\input{C:/Users/khali/OneDrive/AUS/Classes/6 - F23/preamble.tex}

\hypersetup{
	colorlinks=true,
	linkcolor=blue,
	filecolor=magenta,      
	urlcolor=cyan,
	pdftitle={PHY 313 - HW 2},
	pdfpagemode=FullScreen,
}

\begin{document}
	
	\begin{center}
		\hrule
		\vspace{0.4cm}
		{\textbf { \large PHY 313 --- Satellites \& Space Science}}
		\vspace{0.4cm}
	\end{center}
	{\bd{Name:} \ Khalifa Salem Almatrooshi \hspace{\fill} \bd{Due Date:} 24 Sep 2023 \\
		{ \bd{Student Number:}} \ @00090847 \hspace{\fill} \bd{Assignment:} HW 2 \\
		\hrule	

	\section*{Problem 1: }
	\boldmath
	Knowing that the solar energy flux at Earth is $ 1.37 \; kW \, m^{-2} $, \\
	\begin{enumerate}
		\item[(a)] Determine the total luminosity $L$ of the Sun, knowing the measured solar irradiance on Earth $I_E = S_E = \SI{1.37}{\kilo\watt\per\meter\squared}$. How different is the solar irradiance received by a satellite at different altitudes?
		\paragraph{Solution} For the first part, we can backtrack from the intesnity at Earth with the inverse square law to reach the source, that is the Sun. \unboldmath  		

		\[
			I = \frac{Q}{4\pi r^2}
		\]
		\begin{equation*}
			\begin{split}
				Q &= S_E ( 4\pi (1 \; AU)^2 ) \\
				&= ( \SI{1.37}{\kilo\watt\per\meter\squared} ) ( 4\pi ( \num{1.5e11} \, \unit{\meter} )^2 ) \\
				&= \num{3.87e26} \, \unit{\watt} \approx \num{4e26} \, \unit{\watt} = L_{\odot}
			\end{split}
		\end{equation*}
		
		\[
			S(r) = S_E \left( \frac{r_E}{r} \right)^2 ( 1 - A )
		\]
		Take $A=0$ for now.
		\begin{equation*}
			\begin{split}
				S(Geosynchronous) &= ( \SI{1.37}{\kilo\watt\per\meter\squared} ) \left( \frac{ (\num{1.5e11} \, \unit{\meter}) }{ (\num{1.5e11} \, \unit{\meter}) - (\num{3.6e7} \, \unit{\meter}) } \right)^2 \\
				&= \SI{1.37}{\kilo\watt\per\meter\squared} = S_E
			\end{split}
		\end{equation*}
		Clearly, the solar irradiance is practically the same for all altitudes. The only factor that would change that is the albedo of the solar array, with some online sources giving a range from 0.1 to 0.3
		\begin{equation*}
			\begin{split}
				S(A=0.1) &= ( \SI{1.37}{\kilo\watt\per\meter\squared} ) (1 - 0.1) = \num{1.23e3} \, \unit{\kilo\watt\per\meter\squared} \\
				S(A=0.3) &= ( \SI{1.37}{\kilo\watt\per\meter\squared} ) (1 - 0.3) = \num{9.59e2} \, \unit{\kilo\watt\per\meter\squared} \\
			\end{split}
		\end{equation*}

		\boldmath
		\item[(b)] A solar array of area $\SI{3.6}{\meter\squared}$ with an overall solar to electric energy conversion efficiency of 24\% orbits the Earth. Calculate the power generated by the solar array when it faces the Sun and when it makes an angle of $\SI{25}{\degree}$ with the incident light.
		\paragraph{Solution} The basic formula for intensity applies here. \unboldmath
		
		\[
			I = \frac{P}{A}
		\]
		The efficiency and $\cos \theta$ are dimensionless values.
		\begin{equation*}
			\begin{split}
				P &= S_E \cdot A \cdot e \cdot \cos \theta \\
				P(\theta = \ang{0}) &= ( \num{1.37e3} \, \unit{\kilo\watt\per\meter\squared} )  ( \num{3.6} \, \unit{\meter\squared} )  ( \num{0.24} ) ( \cos \ang{0}) = \num{1.18e3} \, \unit{\kilo\watt} \\
				P(\theta = \ang{25}) &= ( \num{1.37e3} \, \unit{\kilo\watt\per\meter\squared} )  ( \num{3.6} \, \unit{\meter\squared} )  ( \num{0.24} ) ( \cos \ang{25}) = \num{1.07e3} \, \unit{\kilo\watt}
			\end{split}
		\end{equation*}
		
	\end{enumerate}
	
	\clearpage
	
	\section*{Problem 2: }
	\boldmath
	Knowing that there are $\SI{100}{}$ million pieces of debris in space, we can assume either that: \\
	\begin{enumerate}[itemsep=0mm]
		\item[i.] They are uniformly distributed between $\SI{100}{\kilo\meter}$ (“top” of the atmosphere) and $\SI{40000}{\kilo\meter}$.
		\item[ii.] They are distributed with a density proportional to $1/h$ between $\SI{100}{\kilo\meter}$ and $\SI{40000}{\kilo\meter}$. \\
	\end{enumerate}
	In each case, determine: \\
	\begin{enumerate}
		\item[(a)] the density of debris at $\SI{400}{\kilo\meter}$, $\SI{1000}{\kilo\meter}$, and $\SI{36000}{\kilo\meter}$;
		\paragraph{Solution} Instead of mass density $\rho$, we have number density $n$. \unboldmath \\ 
		
		\begin{enumerate}
			\item[i.] First case is simple to do.
				Let $N = \num{100e6} \, debris$
				\[
				n = \frac{N}{V}
				\]
				Volume $V$ here is the shell formed between the given altitudes.
				\[
				V = \frac{4\pi}{3} \left[ ( R_{outer} + R_E )^3 - ( R_{inner} + R_E )^3 \right] 
				\]
				\begin{equation*}
					\begin{split}
						V &= \frac{4\pi}{3} \left[ ( ( \num{4e7} \, \unit{\meter} ) + ( \num{6371e3} \, \unit{\meter} ) )^3 - ( ( \num{1e5} \, \unit{\meter} ) + ( \num{6371e3} \, \unit{\meter} ) )^3 \right] \\
						&= \num{4.17e23} \, \unit{\meter\cubed}
					\end{split}
				\end{equation*}
				\[
				n = \frac{N}{V} = \frac{\num{100e6} \, debris}{\num{4.17e23} \, \unit{\meter\cubed}} = \num{2.40e-16} \, \unit{debris \per\meter\cubed}
				\]
				Since it is uniformly distributed, the density of debris will be the same at all altitudes.
				
			\item[ii.] Second case requires more setup.
				\[
					n \propto \frac{1}{h}
				\]
				\[
					n(h) = K \frac{1}{h}
				\]
				One way to find $K$ is to normalize the density function according to $N$, since it is a continuous density function we take a definite integral.
				\[
					N = \int_{R_E + \num{100} \, \unit{\kilo\meter}}^{R_E + \num{40000} \, \unit{\kilo\meter}} n(h) \; dV
				\]
				The differential volume $dV$ here is the shell formed by a differential height. $V = \frac{4}{3} \pi h^3$ becomes $dV = 4 \pi h^2 \; dh$
				\[
					N = \int_{R_E + \num{100} \, \unit{\kilo\meter}}^{R_E + \num{40000} \, \unit{\kilo\meter}} K \frac{1}{h} \, 4\pi h^2 \; dh
				\]
				\[
					N = 4\pi K \int_{R_E + \num{100} \, \unit{\kilo\meter}}^{R_E + \num{40000} \, \unit{\kilo\meter}} h \; dh	
				\]
				\begin{equation*}
					\begin{split}
						N &= 4\pi K \left[ \frac{h^2}{2} \right]^{R_E + \num{40000} \, \unit{\kilo\meter}}_{R_E + \num{100} \, \unit{\kilo\meter}} \\
						\num{100e6} \, debris &= 4\pi K \left[ \num{1.05e15} \, \unit{\meter\squared} \right] \\
						K &= \frac{\num{100e6} \, debris}{4\pi [\num{1.05e15} \, \unit{\meter\squared}]} \\
						K &= \num{7.58e-9} \, \unit{debris \per\meter\squared}
					\end{split}
				\end{equation*}
				\[
					n(h) = \frac{\num{7.58e-9} \, \unit{debris \per\meter\squared}}{h}
				\]
				\[
					n(\num{400} \, \unit{\kilo\meter}) = \frac{\num{7.58e-9} \, \unit{debris \per\meter\squared}}{\num{400} \, \unit{\kilo\meter}} = \num{1.90e-14} \, \unit{debris \per\meter\cubed}
				\]
				\[
					n(\num{1000} \, \unit{\kilo\meter}) = \frac{\num{7.58e-9} \, \unit{debris \per\meter\squared}}{\num{1000} \, \unit{\kilo\meter}} = \num{7.58e-15} \, \unit{debris \per\meter\cubed}
				\]
				\[
					n(\num{36000} \, \unit{\kilo\meter}) = \frac{\num{7.58e-9} \, \unit{debris \per\meter\squared}}{\num{36000} \, \unit{\kilo\meter}} = \num{1.90e-16} \, \unit{debris \per\meter\cubed}
				\]
								
		\end{enumerate}
		
		\clearpage
	
		\boldmath
		\item[(b)] the rate of collisions (per year) of a $\SI{2.5}{\meter} \cross \SI{2.5}{\meter}$ satellite with debris (hint: recall the relation between rate of collisions and cross section).
		\paragraph{Solution} From online sources, the velocity here is the relative velocity between the satellite and debris, which is taken on average as $\num{10} \, \unit{\kilo\meter\per\second}$. \unboldmath
			\[
				\text{Rate of collisions} = n \cdot A_{cross} \cdot v
			\]
		\begin{enumerate}
			\item[i.] Since the density of debris is uniformly distributed we only have one value for the rate of collisions between the given altitudes.
			\[
				\text{Rate of collisions} = (\num{2.40e-16} \, \unit{debris \per\meter\cubed}) (\num{2.5} \, \unit{\meter})^2 (\num{10} \, \unit{\kilo\meter\per\second}) = \num{1.5e-11} \, \unit{collisions \per \second}
			\]
			\[
				\num{1.5e-11} \, \left(\unit{\frac{collisions}{s}} \frac{\num{60}\,\unit{\second}}{\num{1}\,\unit{\minute}} \frac{\num{60}\,\unit{\minute}}{\num{1}\,\unit{\hour}} \frac{\num{24}\,\unit{\hour}}{\num{1}\,\unit{\day}} \frac{\num{365}\,\unit{\day}}{\num{1}\,\unit{\y}}\right) = \num{4.73e-4} \, \unit{collisions \per \y}
			\]
			
			\item[ii.] Since the density of debris depends on h we will have a value for the rate of collisions at each altitude.
			\begin{equation*}
				\begin{split}
					\text{Rate of collisions at } \num{400} \, \unit{km} &= (\num{1.90e-14} \, \unit{debris \per\meter\cubed}) (\num{2.5} \, \unit{\meter})^2 (\num{10} \, \unit{\kilo\meter\per\second}) \\
					&= \num{1.19e-9} \, \unit{collisions \per \second} \\
					&= \num{3.75e-2} \, \unit{collisions \per \y}	
				\end{split}
			\end{equation*}
			\begin{equation*}
				\begin{split}
					\text{Rate of collisions at } \num{1000} \, \unit{km} &= (\num{7.58e-15} \, \unit{debris \per\meter\cubed}) (\num{2.5} \, \unit{\meter})^2 (\num{10} \, \unit{\kilo\meter\per\second}) \\
					&= \num{4.74e-10} \, \unit{collisions \per \second} \\
					&= \num{1.50e-2} \, \unit{collisions \per \y}
				\end{split}
			\end{equation*}
			\begin{equation*}
				\begin{split}
					\text{Rate of collisions at } \num{40000} \, \unit{km} &= (\num{1.90e-16} \, \unit{debris \per\meter\cubed}) (\num{2.5} \, \unit{\meter})^2 (\num{10} \, \unit{\kilo\meter\per\second}) \\
					&= \num{1.19e-11} \, \unit{collisions \per \second} \\
					&= \num{3.75e-4} \, \unit{collisions \per \y}
				\end{split}
			\end{equation*}
		\end{enumerate}
		
	\end{enumerate}
	\clearpage
	\section*{Problem 3: }
	\boldmath
	Let us use the expression for the geomagnetic potential function $\psi$:
	\[
		\psi = \frac{a}{\mu_0} \sum_{l=1}^{\infty} \sum_{m=0}^{l} \left( \frac{a}{r} \right)^{l+1} P^m_l (\cos \theta) (g^m_l \cos m\phi + h^m_l \sin m\phi)
	\]
	but retain only the terms with $l=1$. \\
	\begin{enumerate}
		\item[(a)] Derive the expressions of the components of the magnetic field $B$ in terms of $r$, $\theta$, and $\phi$.
		\paragraph{Solution}   \unboldmath
		
		\[
			\psi(r, \theta, \phi) = \frac{a}{\mu_0} \left[ \left( \frac{a}{r} \right)^{2} P^0_1 (\cos \theta) (g^0_1 \cos ((0)\phi) + h^0_1 \sin ((0)\phi)) + \left( \frac{a}{r} \right)^{2} P^1_1 (\cos \theta) (g^1_1 \cos ((1)\phi) + h^1_1 \sin ((1)\phi)) \right]
		\]
		\[
			\psi = \frac{a^3}{\mu_0 r^2} \left[ ( \cos \theta ) (g^0_1) - (1 - \cos^2 \theta)^{1/2} (g^1_1 \cos \phi + h^1_1 \sin \phi) \right]
		\]
		\[
			\psi = \frac{a^3}{\mu_0 r^2} \left[ g^0_1 \cos \theta - \sin \theta (g^1_1 \cos \phi + h^1_1 \sin \phi) \right]
		\]
		\[
			\psi(r, \theta, \phi) = \frac{a^3}{\mu_0 r^2} \left[ g^0_1 \cos \theta -  g^1_1 \sin \theta \cos \phi - h^1_1 \sin \theta \sin \phi \right]
		\]
		
		The magnetic field $B$ of the Earth is related to it's magnetic potential function $\psi$ in this way because it is a conservative vector field.
		\[
			\vec{B} = -\vec{\nabla} \psi 
		\]
		\[
			\vec{\nabla} = \frac{\partial}{\partial r} \; \hat{r} + \frac{1}{r} \frac{\partial}{\partial \theta} \; \hat{\theta} + \frac{1}{r \sin \theta} \frac{\partial}{\partial \phi} \; \hat{\phi}
		\]
		\begin{equation*}
			\begin{split}
				\hat{r} &: \quad \frac{\partial}{\partial r} \frac{a^3}{\mu_0 r^2} \left[ g^0_1 \cos \theta -  g^1_1 \sin \theta \cos \phi - h^1_1 \sin \theta \sin \phi \right] \\
				&= -\frac{2a^3}{\mu_0 r^3} \left[ g^0_1 \cos \theta -  g^1_1 \sin \theta \cos \phi - h^1_1 \sin \theta \sin \phi \right]
			\end{split}
		\end{equation*}
		
		\begin{equation*}
			\begin{split}
				\hat{\theta} &: \quad \frac{1}{r} \frac{\partial}{\partial \theta} \frac{a^3}{\mu_0 r^2} \left[ g^0_1 \cos \theta -  g^1_1 \sin \theta \cos \phi - h^1_1 \sin \theta \sin \phi \right] \\
				&= \frac{a^3}{\mu_0 r^3} \frac{\partial}{\partial \theta} \left[ g^0_1 \cos \theta -  g^1_1 \sin \theta \cos \phi - h^1_1 \sin \theta \sin \phi \right] \\
				&= \frac{a^3}{\mu_0 r^3} \left[ - g^0_1 \sin \theta -  g^1_1 \cos \theta \cos \phi - h^1_1 \cos \theta \sin \phi \right]
			\end{split}
		\end{equation*}
		
		\begin{equation*}
			\begin{split}
				\hat{\phi} &: \frac{1}{r \sin \theta} \frac{\partial}{\partial \phi} \frac{a^3}{\mu_0 r^2} \left[ g^0_1 \cos \theta -  g^1_1 \sin \theta \cos \phi - h^1_1 \sin \theta \sin \phi \right] \\
				&= \frac{a^3}{\mu_0 r^3} \frac{1}{\sin \theta} \frac{\partial}{\partial \phi} \left[ g^0_1 \cos \theta -  g^1_1 \sin \theta \cos \phi - h^1_1 \sin \theta \sin \phi \right] \\
				&= \frac{a^3}{\mu_0 r^3} \frac{1}{\sin \theta} \left[ 0 + g^1_1 \sin \theta \sin \phi - h^1_1 \sin \theta \cos \phi \right] \\
				&= \frac{a^3}{\mu_0 r^3} \left[ g^1_1 \sin \phi - h^1_1 \cos \phi \right]
			\end{split}
		\end{equation*}
		It is typical when dealing with spherical coordinates to find that components depend on each other, as apparent from $\vec{\nabla}$. Now the final equation for $\vec{B}$:
		\begin{equation*}
			\begin{split}
				\vec{B}(r,\theta,\phi) = &\left[ \frac{2a^3}{\mu_0 r^3} \left[ g^0_1 \cos \theta - g^1_1 \sin \theta \cos \phi - h^1_1 \sin \theta \sin \phi \right] \right] \hat{r} \\
				+ &\left[ \frac{a^3}{\mu_0 r^3} \left[ g^0_1 \sin \theta + g^1_1 \cos \theta \cos \phi + h^1_1 \cos \theta \sin \phi \right] \right] \hat{\theta} \\
				+ &\left[ \frac{a^3}{\mu_0 r^3} \left[ h^1_1 \cos \phi - g^1_1 \sin \phi \right] \right] \hat{\phi}
			\end{split}
		\end{equation*}
		
		\clearpage
		
		\boldmath
		\item[(b)] Knowing the magnetic field (intensity and direction) at several places on the Earth’s surface (equator, poles), determine the values of the $g$ and $h$ coefficients.
		\paragraph{Solution} Using the values from the slides. \unboldmath
		
		\begin{equation*}
			\begin{split}
				\vec{B} = &\left[ \frac{2a^3}{\mu_0 r^3} \left[ g^0_1 \cos \theta - g^1_1 \sin \theta \cos \phi - h^1_1 \sin \theta \sin \phi \right] \right] \hat{r} \\
				+ &\left[ \frac{a^3}{\mu_0 r^3} \left[ g^0_1 \sin \theta + g^1_1 \cos \theta \cos \phi + h^1_1 \cos \theta \sin \phi \right] \right] \hat{\theta} \\
				+ &\left[ \frac{a^3}{\mu_0 r^3} \left[ h^1_1 \cos \phi - g^1_1 \sin \phi \right] \right] \hat{\phi}
			\end{split}
		\end{equation*}
		
		Near the equator $\abs{\vec{B}} = \num{3e-5} \, \unit{\tesla}$ and the magnetic field lines are horizontal, meaning that they are aligned along the surface from pole to pole, and not radially away. Therefore only the $\hat{\theta}$ and $\hat{\phi}$ components will survive. To find the Gauss coefficients for this case at the surface, we can take $r = a$, $\theta = \frac{\pi}{2}$, and $\phi$ would range from $0$ to $2\pi$:
		
		%\begin{equation*}
		%	\begin{split}
		%		\vec{B}(a,\frac{\pi}{2},\phi) = &\left[ \frac{2a^3}{\mu_0 a^3} \left[ g^0_1 \cos \frac{\pi}{2} - g^1_1 \sin \frac{\pi}{2} \cos \phi - h^1_1 \sin \frac{\pi}{2} \sin \phi \right] \right] \hat{r} \\
		%		+ &\left[ \frac{a^3}{\mu_0 a^3} \left[ g^0_1 \sin \frac{\pi}{2} + g^1_1 \cos \frac{\pi}{2} \cos \phi + h^1_1 \cos \frac{\pi}{2} \sin \phi \right] \right] \hat{\theta} \\
		%		+ &\left[ \frac{a^3}{\mu_0 a^3} \left[ h^1_1 \cos \phi - g^1_1 \sin \phi \right] \right] \hat{\phi}
		%	\end{split}
		%\end{equation*}
		%
		%\begin{equation*}
		%	\begin{split}
		%		\vec{B}(a,\frac{\pi}{2},\phi) = \left[ \frac{2}{\mu_0} \left[ -g^1_1 \cos \phi - h^1_1 \sin \phi \right] \right] \hat{r}
		%		+ \left[ \frac{1}{\mu_0} \left[ g^0_1 \right] \right] \hat{\theta}
		%		+ \left[ \frac{1}{\mu_0} \left[ h^1_1 \cos \phi - g^1_1 \sin \phi \right] \right] \hat{\phi}
		%	\end{split}
		%\end{equation*}
		
		\begin{equation*}
			\begin{split}
				\vec{B}\left(a,\frac{\pi}{2},\phi\right) = &\left[ \frac{a^3}{\mu_0 a^3} \left[ g^0_1 \sin \frac{\pi}{2} + g^1_1 \cos \frac{\pi}{2} \cos \phi + h^1_1 \cos \frac{\pi}{2} \sin \phi \right] \right] \hat{\theta} \\
				+ &\left[ \frac{a^3}{\mu_0 a^3} \left[ h^1_1 \cos \phi - g^1_1 \sin \phi \right] \right] \hat{\phi}
			\end{split}
		\end{equation*}
		
		%\begin{equation*}
		%	\begin{split}
		%		\vec{B}\left(a,\frac{\pi}{2},\phi\right) = &\left[ \frac{a^3}{\mu_0 a^3} \left[ g^0_1 \sin \frac{\pi}{2} + g^1_1 \cos \frac{\pi}{2} \cos \phi + h^1_1 \cos \frac{\pi}{2} \sin \phi \right] \right] \hat{\theta} \\
		%	\end{split}
		%\end{equation*}
		%
		%\begin{equation*}
		%	\begin{split}
		%		\vec{B}\left(a,\frac{\pi}{2},\phi\right) = \left[ \frac{1}{\mu_0} \left[ g^0_1 \right] \right] \hat{\theta}
		%	\end{split}
		%\end{equation*}
		
		\begin{equation*}
			\begin{split}
				\vec{B}\left(a,\frac{\pi}{2},\phi\right) = \left[ \frac{1}{\mu_0} \left[ g^0_1 \right] \right] \hat{\theta}
				+ \left[ \frac{1}{\mu_0} \left[ h^1_1 \cos \phi - g^1_1 \sin \phi \right] \right] \hat{\phi}
			\end{split}
		\end{equation*}
		
		We can take $\phi = 0$ or $\frac{\pi}{2}$ to isolate the coefficients.
		
		\begin{equation*}
			\begin{split}
				\vec{B}\left(a,\frac{\pi}{2},0\right) = \left[ \frac{1}{\mu_0} \left[ g^0_1 \right] \right] \hat{\theta}
				+ \left[ \frac{1}{\mu_0} \left[ h^1_1  \right] \right] \hat{\phi}
			\end{split}
		\end{equation*}
		
		\begin{equation*}
			\begin{split}
				\abs{\vec{B}\left(a,\frac{\pi}{2},0\right)} &= \sqrt{\left[ \frac{1}{\mu_0} \left[ g^0_1 \right] \right]^2
				+ \left[ \frac{1}{\mu_0} \left[ h^1_1  \right] \right]^2} \\
				\num{3e-5} \, \unit{\tesla} &= \frac{1}{\mu_0} \sqrt{\left[ g^0_1 \right]^2
				+ \left[ h^1_1 \right]^2} \\
			\end{split}
		\end{equation*}
		
		\begin{equation*}
			\begin{split}
				\vec{B}\left( a,\frac{\pi}{2},\frac{\pi}{2} \right) = \left[ \frac{1}{\mu_0} \left[ g^0_1 \right] \right] \hat{\theta}
				+ \left[ \frac{1}{\mu_0} \left[ -g^1_1  \right] \right] \hat{\phi}
			\end{split}
		\end{equation*}
		
		\begin{equation*}
			\begin{split}
				\abs{\vec{B}\left( a,\frac{\pi}{2},\frac{\pi}{2} \right)} &= \sqrt{\left[ \frac{1}{\mu_0} \left[ g^0_1 \right] \right]^2
					+ \left[ \frac{1}{\mu_0} \left[ -g^1_1  \right] \right]^2} \\
				\num{3e-5} \, \unit{\tesla} &= \frac{1}{\mu_0} \sqrt{\left[ g^0_1 \right]^2
					+ \left[ g^1_1 \right]^2} \\
			\end{split}
		\end{equation*}
				
		Near the poles $\abs{\vec{B}} = \num{6e-5} \, \unit{\tesla}$ and the magnetic field lines are vertical, meaning that they are aligned with the radial direction and not along the surface. Therefore, only the $\hat{r}$ component will survive. To find the Gauss coefficients for this case at the surface, we can take $r = a$, $\theta = 0 \text{ or } \pi$, and $\phi$ would be undefined at the poles but near them it would range from $0$ to $2\pi$:
		
		%\begin{equation*}
		%	\begin{split}
		%		\vec{B}(a,0,\phi) = &\left[ \frac{2a^3}{\mu_0 a^3} \left[ g^0_1 \cos 0 - g^1_1 \sin 0 \cos \phi - h^1_1 \sin 0 \sin \phi \right] \right] \hat{r} \\
		%		+ &\left[ \frac{a^3}{\mu_0 a^3} \left[ g^0_1 \sin 0 + g^1_1 \cos 0 \cos \phi + h^1_1 \cos 0 \sin \phi \right] \right] \hat{\theta} \\
		%		+ &\left[ \frac{a^3}{\mu_0 a^3} \left[ h^1_1 \cos \phi - g^1_1 \sin \phi \right] \right] \hat{\phi}
		%	\end{split}
		%\end{equation*}
		%
		%\begin{equation*}
		%	\begin{split}
		%		\vec{B}(a,0,\phi) = \left[ \frac{2}{\mu_0} \left[ g^0_1 \right] \right] \hat{r}
		%		+ \left[ \frac{1}{\mu_0} \left[ g^1_1 \cos \phi + h^1_1 \sin \phi \right] \right] \hat{\theta}
		%		+ \left[ \frac{1}{\mu_0} \left[ h^1_1 \cos \phi - g^1_1 \sin \phi \right] \right] \hat{\phi} 
		%	\end{split}
		%\end{equation*}
		
		\begin{equation*}
			\begin{split}
				\vec{B}(a,0,\phi) &= \left[ \frac{2a^3}{\mu_0 a^3} \left[ g^0_1 \cos 0 - g^1_1 \sin 0 \cos \phi - h^1_1 \sin 0 \sin \phi \right] \right] \hat{r} \\
				&= \left[ \frac{2}{\mu_0} g^0_1 \right] \hat{r} \\
			\end{split}
		\end{equation*}
		
		\begin{equation*}
			\begin{split}
				\abs{\vec{B}\left(a,0,\phi\right)} &= \sqrt{\left[ \frac{2}{\mu_0} g^0_1 \right]^2 } \\
				\num{6e-5} \, \unit{\tesla} &= \frac{2}{\mu_0} g^0_1
			\end{split}
		\end{equation*}
		
		Now we have everything we need to find the Gauss coefficients. $\mu_0 = 4\pi \cross 10^{-7} \, \unit{\tesla \cdot \meter\per\ampere}$
		
		\begin{equation*}
			\begin{split}
				g^0_1 &= \mu_0 ( \num{3e-5} \, \unit{\tesla} ) \\
				&= \num{3.77e-11} \, \unit{\tesla^2 \cdot \meter \per \ampere}
			\end{split}
		\end{equation*}
		
		The $\mu_0$ here can be ignored to find that the value agrees with online values for $g^0_1 \approx \num{2.98e-5} \, \unit{\tesla}$.
		
		\begin{equation*}
			\begin{split}
				\num{3e-5} \, \unit{\tesla} &= \frac{1}{\mu_0} \sqrt{\left[ g^0_1 \right]^2	+ \left[ h^1_1 \right]^2} \\
				%\num{3e-5} \, \unit{\tesla} &= \frac{1}{\mu_0} \sqrt{\left[ \mu_0 ( \num{3e-5} \, \unit{\tesla} ) \right]^2	+ \left[ h^1_1 \right]^2} \\
				%\num{3e-5} \, \unit{\tesla} &= \frac{1}{\mu_0} \sqrt{\left[ \mu_0 ( \num{3e-5} \, \unit{\tesla} ) \right]^2	+ \left[ h^1_1 \right]^2} \\
			\end{split}
		\end{equation*}
		
		\begin{equation*}
			\begin{split}
				\num{3e-5} \, \unit{\tesla} &= \frac{1}{\mu_0} \sqrt{\left[ g^0_1 \right]^2	+ \left[ g^1_1 \right]^2} \\
			\end{split}
		\end{equation*}
		
		I was not able to find the rest of the coefficients. I would think my setup by dropping components is the problem. Some sources "absorb" the minus sign from $P^1_1(\cos \theta)$ into the Gauss coefficients, which could affect the result?
		
		
	\end{enumerate}
	
	\section*{Problem 4: }
	\boldmath
	\begin{enumerate}
		\item[(a)] Calculate the transfer time to Mars in a Hohmann trajectory.
		\paragraph{Solution} The Hohmann trajectory is the half ellipse formed where the perihelion is at the orbit of the initial body, and the aphelion is at the orbit of the destination body. The semi-major axis of the elliptical orbit can be used with Kepler's Third Law to find the transfer time. \\ \unboldmath
		
		The semi-major axis is half the distance from the perihelion to the aphelion. and that distance is the sum of the distances from the Sun to Earth and the Sun to Mars.
		\[
			a = \frac{r_E + r_{Mars}}{2}
		\]
		\[
			a = \frac{(\num{1.5e11} \, \unit{\meter}) + (\num{2.28e11} \, \unit{\meter})}{2} = (\num{1.89e11} \, \unit{\meter})
		\]
		\[
			T^2 = \frac{4\pi^2}{GM_{\bigodot}} a^3
		\]
		The elliptical orbit has the sun at one focus, and the ellipse itself is formed by the gravitational pull of the Sun.
		\[
			T = \sqrt{\frac{4\pi^2 (\num{1.89e11} \, \unit{\meter})^3}{(\num{6.6743e-11} \, \unit{\newton\meter\squared\per\kilogram\squared})(\num{2e30} \, \unit{\kilogram})}} = \num{4.47e7} \, \unit{\second}
		\]
		Since the Hohmann trajectory is only half of the elliptical orbit, we can take half of the orbital period to find the transfer time.
		\[
			t = \frac{\num{4.47e7} \, \unit{\second}}{2} = \num{2.235e7} \, \unit{\second} \left( \frac{\num{1} \, \unit{\minute}}{\num{60} \, \unit{\second}} \frac{\num{1} \, \unit{\hour}}{\num{60} \, \unit{\minute}} \frac{\num{1} \, \unit{\day}}{\num{24} \, \unit{\hour}} \right) \approx \num{259} \, \unit{\day}
		\]
		
		\boldmath
		\item[(b)] Using the daily motions of Earth and Mars, compute the ideal relative position of Earth and Mars during launch. To do that, consider: how much angular distance Mars moves in 1 day, and where Mars must be relative to Earth at launch for Rendez-Vous at Mars aphelion. 
		\paragraph{Solution} To find how much angular distance mars moves in 1 day, we need to divide $\ang{360}$ by a martian year in days. \\ \unboldmath
		\[
			\dot{\theta}_{Mars} = \frac{\ang{360}}{\num{687} \, \unit{\day}} = \num{0.986} \, \unit{\degree\per\day}
		\]
		We can use this to backtrack from the orbital aphelion, the destination, to find and Mars's initial position.
		\[
			\theta_{Mars} = (\num{259} \, \unit{\day})(\num{0.524} \, \unit{\degree\per\day}) = \ang{136}
		\]
		\[
			\theta_{rel} = \ang{180} - \ang{136} = \ang{44}
		\]
		Mars is ahead of Earth by $\ang{44}$ when the spacecraft is injected into the Hohmann orbit perihelion.
		
		\clearpage
		
		\boldmath
		\item[(c)] Relative to Mars, where is Earth in its orbit when the spacecraft arrives? 
		\paragraph{Solution} Similar to the previous part. \unboldmath
		\[
			\dot{\theta}_{Earth} = \frac{\ang{360}}{\num{365} \, \unit{\day}} = \num{0.986} \, \unit{\degree\per\day}
		\]
		We can use this to predict from the orbital perihelion, the origin, Earth's position when the spacecraft arrives.
		\[
			\theta_{Earth} = (\num{259} \, \unit{\day})(\num{0.986} \, \unit{\degree\per\day}) = \ang{255}
		\]
		\[
			\theta_{rel} = \ang{255} - \ang{180} = \ang{75}
		\]
		Earth is ahead of Mars by $\ang{75}$ when the spacecraft arrives at the orbital aphelion, the intersect with Mar's orbit.
		
		
		\boldmath
		\item[(d)] What is the periodicity of this kind of launch to Mars? 
		\paragraph{Solution} We have to find when the optimal configuration from part b occurs again. We know that Earth is ahead of Mars by $\ang{75}$ when the spacecraft arrives. Now we need to find when Mars will be $\ang{44}$ ahead of Earth. By taking the relative $\dot{\theta}$ we can get the changing angular distance that the Sun see's everyday between Earth and Mars. \unboldmath
		\[
			\dot{\theta_{rel}} = \num{0.986} \, \unit{\degree\per\day} - \num{0.524} \, \unit{\degree\per\day} = \num{0.462} \, \unit{\degree\per\day}
		\]
		This will help to find the synodic period for Earth to return to the same relative position with Mars. This can extend to any orbit configuration.
		\[
			T = \frac{\ang{360}}{\num{0.462} \, \unit{\degree\per\day}} = \num{779} \, \unit{\day} \approx \num{26} \, \unit{months}
		\]
		
		
		\item[(e)] Must a spacecraft be launched at an exact moment (in such arrangements)? What happens if the spacecraft is launched early or late? What is the average length of a launch window to Mars (find some reference for this info)? 
		\paragraph{Solution} What we have found is the launch period, the time between launch opportunities. Fortunately, we also have launch windows, where it is possible to launch earlier or later on the same day depending on many factors, environmental, technical, etc. They exist because we can compensate mid flight to reach the optimal orbit. For the Emirates Mars Mission, the Mars Hope orbiter was launched in Japan during a launch window starting from 16 July 2020 through to August 13 2020 (\url{https://www.mhi.com/notice/notice_200518.html}). Also, this source has launch windows from 2007 to 2020, some are 2 months and others are 3 months. it is important to note that they state "These data assume purely ballistic trajectories and
		should be used for preliminary analyses", so compensation mid-flight is not accounted for. (\url{https://mepag.jpl.nasa.gov/reports/3715_Mars_Expl_Strat_GPO.pdf})
		
		
		
	\end{enumerate}
	
	
\end{document}