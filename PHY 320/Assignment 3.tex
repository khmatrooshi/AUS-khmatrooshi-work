\documentclass[]{article}

\usepackage[margin=1.0in]{geometry}
\usepackage{amsmath, amsfonts, amssymb, amsthm}
\usepackage{bbold}
\usepackage{graphicx, wrapfig}
\usepackage{tikz}
\usepackage{titling}
\usepackage{siunitx}


\setlength{\droptitle}{-10em}

\title{PHY 320 - Assignment 3}
\author{Khalifa Salem Almatrooshi b00090847}
\date{09/10/2022}

\newcommand{\bd}{\textbf}
\newcommand{\ih}{\bd{i}}
\newcommand{\jh}{\bd{j}}
\newcommand{\kh}{\bd{k}}
\newcommand{\ehr}{\hat{e}_r}
\newcommand{\ehth}{\hat{e}_\theta}

\begin{document}
	
	\maketitle
	
	\section{Problem 3}
	
	A particle undergoes simple harmonic motion with a frequency of 10 Hz. Find the displacement $ x $ at any time $ t $ for the following initial conditions: 
	\begin{equation}
		\begin{split}
			t = 0, \ x = 0.25 \, \si{m}, \ \dot{x} = 0.1 \, \si{m/s} \\
		\end{split}
	\end{equation}

	\bd{Solution} Another way to solve homogeneous differential equations is through the auxiliary polynomial. Where the complex root $ m = \alpha \pm \beta i $ corresponds to $ x(t) = e^{\alpha t}\left( A\cos(+\beta t) + B\sin(+\beta t) \right) $.
	
	\begin{equation}
		\begin{split}
			\omega &= 2\pi f = 20\pi \\
			m\ddot{x} + kx &= 0 \\
			\ddot{x} + \omega^2 x &= 0 \\
			a(r) = r^2 + \omega^2 &= 0 \\
			r &= 0 \pm \omega i \\
			x(t) &= A\cos(\omega t) + B\sin(\omega t) \\
			\dot{x}(t) &= -A \omega \sin(\omega t) + B \omega \cos(\omega t) \\
		\end{split}
	\end{equation}

	\begin{equation}
		\begin{split}
			x(0) = A + 0 = 0.25 \\
			A = \frac{1}{4} \\
			\dot{x}(0) = 0 + B\omega = 0.1 \\
			B = \frac{0.1}{\omega} = \frac{1}{200\pi} \\
		\end{split}
	\end{equation}

	\begin{equation}
		\begin{split}
			x(t) &= \frac{1}{4}\cos(20\pi t) + \frac{1}{200\pi}\sin(20\pi t) \\
		\end{split}
	\end{equation}

	
	\section{Problem 5}
	
	A particle undergoing simple harmonic motion has a velocity $ \dot{x}_1 $ when the displacement is $ x_1 $ and a velocity $ \dot{x}_2 $ when the displacement is $ x_2 $. Find the angular frequency and the amplitude of the motion in terms of the given quantities. \\
	
	\bd{Solution} Through the conservation of mechanical energy, $ T_0 + V_0 = T + V $. 
	\begin{equation}
		\begin{split}
			T_1 + V_1 &= T_2 + V_2 \\
			\frac{1}{2} m \dot{x}_1^2 + \frac{1}{2} k x_1^2 &= \frac{1}{2} m \dot{x}_2^2 + \frac{1}{2} k x_2^2 \\
			m \dot{x}_1^2 + k x_1^2 &= m \dot{x}_2^2 + k x_2^2 \\
			m \left( \dot{x}_1^2 - \dot{x}_2^2 \right) &= k \left( x_2^2 - x_1^2 \right) \\
			\sqrt{\frac{k}{m}} &= \left( \frac{\dot{x}_1^2 - \dot{x}_2^2} {x_2^2 - x_1^2}  \right)^\frac{1}{2} = \omega \\
		\end{split}
	\end{equation}

	\begin{equation}
		\begin{split}
			A^2 &= x_1^2 + \frac{\dot{x}_1^2}{\omega^2} \\
			A^2 &= x_1^2 + \dot{x}_1^2 \left( \frac{x_2^2 - x_1^2} {\dot{x}_1^2 - \dot{x}_2^2} \right) \\
			A &= \left( x_1^2 + \frac{x_2^2 \dot{x}_1^2 - x_1^2 \dot{x}_1^2} {\dot{x}_1^2 - \dot{x}_2^2} \right)^\frac{1}{2} \\
			A &= \left( \frac{ x_1^2 \dot{x}_1^2 - x_1^2 \dot{x}_2^2 + x_2^2 \dot{x}_1^2 - x_1^2 \dot{x}_1^2} {\dot{x}_1^2 - \dot{x}_2^2} \right)^\frac{1}{2} \\
			A &= \left( \frac{ x_2^2 \dot{x}_1^2 - x_1^2 \dot{x}_2^2 } {\dot{x}_1^2 - \dot{x}_2^2} \right)^\frac{1}{2} \\
		\end{split}
	\end{equation}

	\section{Problem 11}
	
	A mass $ m $ moves along the x-axis subject to an attractive force given by $ 17 \beta^2 m x / 2 $ and a retarding force given by $ 3 \beta m \dot{x} $, where $ x $ is its distance from the origin and $\beta$ is a constant. A driving force given by $ m A \cos \omega t $, where A is a constant, is applied to the particle along the x-axis.
	\\\\
	\bd{(a)} What value of $ \omega $ results in steady-state oscillations about the origin with maximum amplitude? \\
	
	\bd{Solution} As the equation of motion seems to be a damped system, the value for omega to result in steady-state oscillations is the resonant frequency of the system. $ \omega_r^2 = \omega_0^2 + 2\gamma^2 $.
	\begin{equation}
		\begin{split}
			m\ddot{x} = -\frac{17}{2}\beta^2 mx - 3\beta m\dot{x} + mA\cos{\omega t} \\
			m\ddot{x} + 3\beta m\dot{x} + \frac{17}{2}\beta^2 mx &= mA\cos{\omega t} \\
			\gamma &= \frac{3\beta m}{2m} = \frac{3\beta}{2} \\
			\omega_0^2 &= \frac{17\beta^2 m}{2m} = \frac{17\beta^2}{2} \\
			\omega_r^2 &= \frac{17\beta^2}{2} + 2\left( \frac{3\beta}{2} \right)^2 = 4\beta^2 \\
			\omega_r &= 2\beta \\
		\end{split}
	\end{equation}
	\\
	\bd{(b)} What is the maximum amplitude? \\

	\bd{Solution} Using the equation for amplitude as a function of angular frequency, with an input of the resonant frequency of the system to get the maximum amplitude.
	\begin{equation}
		\begin{split}
			A(\omega) &= \frac{F_0 / m}{\sqrt{ \left( \omega_0^2 - \omega^2 \right)^2 + \left( 2\gamma \omega \right)^2 }} \\
			A(2\beta) &= \frac{F_0 / m}{\sqrt{ \left( \frac{17\beta^2}{2} - 4\beta^2 \right)^2 + \left( 2 \left( \frac{3\beta}{2} \right) \left( 2\beta \right) \right)^2 }} \\
			&= \frac{F_0 / m}{\sqrt{ \frac{81\beta^4}{4} + 36\beta^4 }} \\
			&= \frac{F_0 / m}{\frac{15\beta^2}{2}} = \frac{2F_0}{15\beta^2 m} \\
		\end{split}
	\end{equation}

	\section{Problem 13}
	
	Given: The amplitude of a damped harmonic oscillator drops to $ 1 / e $ of its initial value after $ n $ complete cycles. Show that the ratio of period of the oscillation to the period of the same oscillator with no damping is given by
	\begin{equation}
		\begin{split}
			\frac{T_d}{T_0} = \left( 1 + \frac{1}{4 \pi^2 n^2} \right)^\frac{1}{2} \approx 1 + \frac{1}{8 \pi^2 n^2} \\
		\end{split}
	\end{equation}

	\bd{Solution} According to the Analytical mechanics book in page 101, for a damped harmonic oscillator: "In one complete period the amplitude diminishes by a factor $ e^{-\gamma T_d} $ ".
	
	\begin{equation}
		\begin{split}
			x(t) &= e^{-\gamma t} A \sin(\omega_d t +\phi_0) \\
			x(T_d) &= e^{-\gamma T_d} A \sin(\omega_d T_d +\phi_0) \\
			x(T_d) &= e^{-\gamma T_d} A \sin(\phi_0) = e^{-\gamma T_d} A \\
			A\left(  e^{-\gamma T_d} \right)^n &= Ae^{-1} \\
			e^{-\gamma T_d n}  &= e^{-1} \\
			\gamma T_d n &= 1 \\
			\gamma &= \frac{1}{T_d n} = \frac{\omega_d}{2\pi n} \\
		\end{split}
	\end{equation}
	Now with a value for gamma in terms of $ \omega_d $: $ \omega_d^2 = \omega_0^2 - \gamma^2 $.
	\begin{equation}
		\begin{split}
			\omega_0 &= \left( \omega_d^2 + \gamma^2 \right)^{1/2} = \left( \omega_d^2 + \frac{\omega_d^2}{4\pi^2 n^2} \right)^{1/2} \\
			&= \omega_d \left( 1 + \frac{1}{4\pi^2 n^2} \right)^{1/2} \\
			\frac{T_d}{T_0} &= \frac{\frac{2\pi}{w_d}}{\frac{2\pi}{w_0}} = \frac{\omega_0}{\omega_d} = \frac{\omega_d \left( 1 + \frac{1}{4\pi^2 n^2} \right)^{1/2}}{\omega_d} = \left( 1 + \frac{1}{4\pi^2 n^2} \right)^{1/2} \\
		\end{split}
	\end{equation}
	As n gets larger, $ \frac{1}{4\pi^2 n^2} $ gets smaller. A useful approximation is given in Appendix D.
	\begin{equation}
		\begin{split}
			(1 + x)^{1/2} &\approx 1 + \frac{1}{2} x \\
			\left( 1 + \frac{1}{4\pi^2 n^2} \right)^{1/2} &\approx 1 + \frac{1}{8\pi^2 n^2} \\
		\end{split}
	\end{equation}
	
	\section{Problem 18}
	
	Solve the differential equation of motion of the damped harmonic oscillator driven by a damped harmonic force: (Hint: $ e^{-\alpha t} \cos \omega t = Re(e^{-\alpha t + i \omega t}) = Re(e^{\beta t}) $, where $ \beta = -\alpha + i \omega $. Assume a solution of the form $ Ae^{\beta t - i \phi} $)
	\begin{equation}
		\begin{split}
			F_{ext}(t) = F_0 e^{-\alpha t} \cos \omega t  \\		
		\end{split}
	\end{equation}

	\bd{Solution} As a general approach to solving differential equations: Assuming a solution of the form $ Ae^{\beta t - i \phi} $, which will give the particular solution $ x_p(t) $ of the general solution; generally referred to as the steady state solution. However, the new concept here is a harmonic driving force that is damped, which means both terms will be transient due to the limiting exponential. Using the hint.
	
	\begin{equation}
		\begin{split}
			F_{ext}(t) = Re(F_0 e^{\beta t}) \\
		\end{split}
	\end{equation}
	Now by demanding the real part of the equation of motion of the damped harmonic oscillator. Following section 3.6 in the book.
	\begin{equation}
		\begin{split}
			m\ddot{x} + b\dot{x} + kx = F_0 e^{\beta t} \\
			x(t) = x_c(t) + x_p(t) \\
		\end{split}
	\end{equation}
	\begin{equation}
		\begin{split}
			x_p(t) = Ae^{\beta t - i \phi} \\
			\dot{x}_p(t) = A\beta e^{\beta t - i \phi} \\
			\ddot{x}_p(t) = A\beta^2 e^{\beta t - i \phi} \\
		\end{split}
	\end{equation}
	\begin{equation}
		\begin{split}
			mA\beta^2 e^{\beta t - i \phi} + bA\beta e^{\beta t - i \phi} + kAe^{\beta t - i \phi} &= F_0 e^{\beta t} \\
			m\beta^2 + b\beta + k &= \frac{F_0}{A} e^{i \phi} \\
			m(-\alpha + i \omega)^2 + b(-\alpha + i \omega) + k &= \frac{F_0}{A} (\cos \phi + i\sin \phi) \\
			m\alpha^2 - 2im\alpha \omega - m\omega^2 - b\alpha + ib\omega + k &= \frac{F_0}{A} \cos \phi + \frac{F_0}{A} i\sin \phi \\
		\end{split}
	\end{equation}
	Equating the real and imaginary parts of the equation.
	\begin{equation}
		\begin{split}
			m\alpha^2 - m\omega^2 - b\alpha + k &= \frac{F_0}{A} \cos \phi \\
			- 2m\alpha \omega + b\omega &= \frac{F_0}{A} \sin \phi \\
		\end{split}
	\end{equation}
	2 equations and 2 unknowns. $ \phi $ from $ \frac{\sin \phi}{\cos \phi} = \tan \phi $, and A from $ \sin^2 +\cos^2 = 1 $.
	\begin{equation}
		\begin{split}
			\frac{\frac{F_0}{A} \sin \phi}{\frac{F_0}{A} \cos \phi} &= \frac{b\omega - 2m\alpha \omega}{m\alpha^2 - m\omega^2 - b\alpha + k} \\
			\phi &= \arctan \left( \frac{b\omega - 2m\alpha \omega}{m\alpha^2 - m\omega^2 - b\alpha + k} \right) \\
		\end{split}
	\end{equation}
	\begin{equation}
		\begin{split}
			\left( \frac{F_0}{A} \sin \phi \right)^2 + \left( \frac{F_0}{A} \cos \phi \right)^2 &= (m\alpha^2 - m\omega^2 - b\alpha + k )^2 + ( b\omega - 2m\alpha \omega)^2 \\
			\frac{F_0^2}{A^2} ( \sin^2 \phi + \cos^2 \phi) &= (m\alpha^2 - m\omega^2 - b\alpha + k )^2 + ( b\omega - 2m\alpha \omega)^2 \\
			A &= \frac{F_0}{ \left( (m\alpha^2 - m\omega^2 - b\alpha + k )^2 + ( b\omega - 2m\alpha \omega)^2 \right)^{1/2} } \\
		\end{split}
	\end{equation}
	Finally with these expressions for $ \phi $ and A.
	\begin{equation}
		\begin{split}
			Re(x_p(t)) &= Re(Ae^{\beta t - i \phi}) = Ae^{-\alpha t + i \omega t - i \phi} = Ae^{-\alpha t} e^ {i(\omega t - \phi)} \\
			&= Ae^{-\alpha t} ( \cos ( \omega t - \phi ) + i\sin ( \omega t - \phi ) ) \\
			x_p(t) &= Ae^{-\alpha t} \cos(\omega t - \phi) \\
		\end{split}
	\end{equation}
	Another transient term would come from solving the homogeneous part of the equation of motion for the complementary solution, which should just be the general solution for a damped harmonic oscillator.
	\begin{equation}
		\begin{split}
			x(t) &= Ae^{-\alpha t} \cos(\omega t - \phi) + x_c(t) \\
			x(t) &= A_pe^{-\alpha t} \cos(\omega_p t - \phi_p) + e^{-\gamma t} \left( A_1 e^{\sqrt{ \gamma^2 - \omega_0^2 } t} + A_2 e^{-\sqrt{ \gamma^2 - \omega_0^2 } t} \right) \\
		\end{split}
	\end{equation}

\end{document}
